\label{tpscm}
\begin{chapterbox}
\vspace{-60pt}
\chapter{Theoretical Physics of Soft Condensed Matter}
\vspace{-30pt}
\centering\normalsize\textit{Lent Term 2018 - Professor M. Cates}
\end{chapterbox}
\vspace{20pt}
%\begin{multicols*}{2}
\minitoc
\newpage
\section{Introduction \& Statics}
What sorts of matter do we consider in this course?
\begin{center}
\begin{mytable}{lll}
& \textbf{Low Tech} & \textbf{High Tech}				\\ \midrule
\emph{Emulsions} & Mayo & Pharma					\\
\emph{Suspensions} & Toothpaste & Paints, Ceramics		\\
\emph{Liquid Crystals} & Slime & Displays				\\
\emph{Polymers} & Chewing Gum & Plastics			
\end{mytable}
\captionof{table}{Simple examples of the types of materials covered in this course}
\end{center}
The \emph{soft} part of the title refers to the characteristic shear modulus\index{modulus!shear};
\begin{equation}
G \sim 10^2 \text{-} 10^7 \,\,\text{Pa}
\end{equation}
which we can compare to something like steel; $G \sim 10^10 \,\,\text{Pa}$. Furthermore, in most cases (except foam\index{foam}), the bulk modulus\index{modulus!bulk},  $\kappa \sim 10^10 \,\,\text{Pa}$, such that in practice $\tfrac{\kappa}{G}\rightarrow \infty$. As such, our materials are \emph{incompressible}\index{incompressible}. They are characteristic of a slow response (order of seconds) to changing conditions e.g. viscoelasticity. The cause of this is large structures within the materials with characteristic scales such as;
\begin{enumerate}
\item polymers\index{polymer}: $\sim 100\text{nm}$
\item colloids\index{colloid}: $\sim 1\mu\text{m}$ 
\item liquid crystal displays\index{LCD}: $\sim 1\mu\text{m}$ 
\end{enumerate}
Furthermore, we won't consider quantum fluctuations, since the time scales on which the occur are negligible in comparison to the slow response discussed above. Thermal fluctuations on the other do matter greatly and will be the major driver of phenomena.

\paraskip
The basic structure of the process starts from a microscopic system. In the \emph{static} case this is coarse grained to a general order parameter $\psi(\vec r)$. At this point one can apply equilibrium statistical mechanics, along with symmetry and conservation laws giving a probability distribution over the spatial order parameter, $\mathbb{P}\left[\psi(\vec r)\right]$. On the other hand, in the \emph{dynamic} picture, which will be our focus, we should start with a hydrodynamic\index{hydrodynamic} description of the phenomena, giving PDEs $\dot{\psi}(\vec r, t) = \cdots$. To get a full description of the dynamics, we promote these PDEs to stochastic PDEs\index{stochastic}, which include noise;
\begin{equation}
\dot{\psi}(\vec r, t) = \cdots + \text{ noise}
\end{equation}
This results in a probability distribution over the possible \emph{evolutions} of the system, $\mathcal{P}\left[\psi(\vec r, t)\right]$.\footnote{We have neglected the very important Fluctuation Dissipation Theorem (FDT)\index{fluctuation dissipation theorem} in this overview. This often fixes the noise in almost al cases using the static probability distribution.}

\paraskip
Now we consider some examples of order parameters in our systems;
\begin{enumerate}
\item density $\rho$, velocity $\vec v$, describing a one component isothermal fluid
\item $\rho$, $\vec v$, composition $\phi$, describing a two component mixture
\item $\rho$, $\vec v$, $\phi$, molecular orientation $\underline{\underline{Q}}$, describing a nematic\index{nematic liquid crystal} liquid crystal
\end{enumerate}
whilst examples of the hydrodynamic equations are;
\begin{itemize}
\item An isothermal, incompressible, one component fluid has $\dot{\rho} = 0$ and $\nabla \cdot \vec v = 0$. This leads to the Navier-Stokes equation\index{equation!Navier-Stokes};
\begin{equation}
\rho\left(\dot{\vec v} + \vec v \cdot \nabla \vec v\right) = \eta \nabla^2 \vec v - \nabla P
\end{equation}
We want to promote this to a stochastic PDE, known as the \emph{Navier-Stokes Landau Lifschitz equation}, which adds a random stress, $\nabla \cdot \underline{\underline{\Sigma}}^N$. The FDT allows us to write down the correlation function of this noise;
\begin{equation}
\langle \Sigma^N_{ij}(\vec r, t) \Sigma^N_{kl}(\vec r, t) \rangle = 2kT \delta(\vec r - \vec r\pr)\delta(t - t\pr)\left(\delta_{ik}\delta_{jl} + \delta_{il}\delta_{jk}\right)
\end{equation}
\end{itemize}
\subsection{Statistical Physics Review}
We take the Gibbs approach to Statistical physics, working from the entropy;
\begin{equation}
S = -k \sum_i{p_i \log p_i}
\end{equation}
$i$ is a microstate\index{microstate} that is a complete description of the microscopics e.g. the positions and momenta of all the molecules etc. Then Gibbs' axiom is that;
\begin{examplebox}
A system in thermal equilibrium maximises $S$ subject to applicable constraints
\end{examplebox}
To make this more precise consider the following examples;
\begin{enumerate}
\item An isolated system fixes $N, E, V$. Then maximise $S$ subject to $\sum_i{p_i} = 1$.\footnote{Note that the sum is performed only over those microstates with fixed $N, E, V$}
\begin{equation}
\Rightarrow p_i = \text{ const.}
\end{equation}
\item A system of fixed contents and volume ($N,V$) in contact with a heat bath. Then $E$ can fluctuate about some average $\langle E \rangle = \bar{E}$. Now the sum is taken over all possible energies with the constraint;
\begin{equation}
\frac{\del}{\del p_i}\left(-k\sum_i{p_i \log p_i} - \lambda_1 \sum_i{p_i} - \lambda_2 \sum_i{p_i E_i}\right) = 0
\end{equation}
This gives simply;
\begin{equation}
p_i = \frac{\exp\left(-\beta E_i\right)}{Z}, \qquad Z = \sum_i{e^{-\beta E_i}}
\end{equation}
\end{enumerate}
We can make contact with thermodynamics; the first law\index{thermodynamics!first law} is $\ud E = T\ud S - p\ud V + \mu \ud N + \,\text{other work terms}$. Then we can read off;
\begin{equation}
\left.\frac{\del S}{\del E}\right|_{V, N, \ldots} = \frac{1}{T}
\end{equation}
Applying Gibb's principle as in the second case, we have;
\begin{equation}
\max\left(S - \lambda_1\sum{p_i} - \lambda_2 \bar{E}\right) \Rightarrow \frac{\del S}{\del \bar{E}} - \lambda_2 = 0 \Rightarrow \beta k = \frac{1}{T}
\end{equation}
So we recover the interpretation of $\beta$ as inverse temperature as is standard. Now define the \emph{Helmholtz free energy}\index{free energy!Helmholtz}
\begin{align}
F(T, V, N) &= \bar{E} - TS = -kT \log Z \\
\ud F &= -S \ud T -p\ud V + \mu \ud N + \cdots
\end{align}
from which we can read off the various partial derivatives that define the evolution of an isothermal system.
\subsection{Coarse Graining\index{coarse graining}}
We want a middle ground between the microstate description of a system and the macroscopic, global variables e.g. $p, T, V$ etc. We define a \emph{mesostate}\footnote{This isn't standard terminology.} as anything that satisfies this middle ground described by some parameter $\psi(\vec r) = \rho(\vec r), \vec v, \phi, \ldots$. Note that inherently a large number of microstates, after coarse graining will realise the same $\psi(\vec r)$. Then we can define the \emph{restricted partition function}\index{partition function!restricted};
\begin{equation}
Z\left[\psi(\vec r)\right] = \exp\left(\beta F[\psi(\vec r)]\right) = \sum_{i \in \psi}{e^{-\beta E_i}}
\end{equation}
where the sum is taken over only those microstates that lead to the same coarse grained description, $\psi$. Writing $F[\psi] = E[\psi] - TS[\psi]$ we have;
\begin{itemize}
\item $E[\psi] = \sum_{i \in \psi}{E_i p_i}$
\item $S[\psi] = kT\sum_{i \in \psi}{p_i \log p_i}$
\end{itemize}
Then the Boltzman distribution\index{Boltzman distribution} becomes;
\begin{equation}
P[\psi] = \sum_{i \in \psi}{p_i} = \frac{e^{-\beta F[\psi]}}{Z_{\text{TOT}}}
\end{equation}
Here $F[\psi]$ is known as the \emph{effective Hamiltonian}\index{Hamiltonian!effective}. Here the normalisation is our partition function $Z_{\text{TOT}}$, given by;
\begin{equation}
Z_{\text{TOT}} = \int{\mathcal{D}[\psi]\,\,e^{-\beta F[\psi]}} \Rightarrow \int{\mathcal{D}[\psi]\,\,P[\psi]} = 1
\end{equation}
So that $F_{\text{TOT}}(T, V, N, \ldots) = -kT\log Z_{\text{TOT}}$. So how do we compute $F[\psi]$? We have two alternatives;
\begin{enumerate}
\item Explicit coarse graining, for example, an ideal gas in an external potential $U_{\text{ext}}(\vec r)$\footnote{This is example $1$.$3$ in the problem sets.}
\item Phenomenological: we can use symmetry\index{symmetry}, conservation laws, differentiability etc. to constrain the form of $F[\psi]$.
\end{enumerate}
As an example of this second method, consider a binary fluid mixture\index{mixture!binary} of $A$ and $B$ molecules. We assume the two fluids to be miscible at high $T$ and immiscible\index{(im)miscible} at low $T$.\footnote{This occurs because at high temperatures, entropy dominates, and the fluids mix, whilst at low temperatures, energy wins and the extra repulsion introduced by $U_{AB}(\vec r)$ ensures the fluids don't mix.} Furthermore assume we have the interactions;
\begin{align*}
\text{same interactions,} \, U_{AA}(\vec r) = U_{BB}(\vec r)&\quad\begin{cases}A\text{-}A \\ B\text{-}B\end{cases} \\
\text{has extra repulsions,} \, U_{AB}(\vec r) \neq U_{AA}(\vec r)&\quad\begin{cases}A\text{-}B\end{cases}	
\end{align*}
We also define the composition variable;
\begin{equation}
\bar{\phi} = \frac{N_A - N_B}{N_A + N_B}
\end{equation}
\begin{mygraphic}{tpscm/binfluid}{0.7}{Inside the binodal curve, we see phase separation\index{phase separation}, where the fluid splits into two regions of volumes $V_{1, 2}$ with different compositions, $\bar{\phi}_{1,2}(T)$ such that $V \bar{\phi} = V_1 \bar{\phi}_1 + V_2\bar{\phi}_2$. Inside the spinodal curve, the uniform phase is locally unstable. Small fluctuations grow. Between the curves however, we see a different behaviour. Starting at $\bar{\phi} = \,\text{const.}$ at high $T$, and dropping the temperature to $T < T_c$ we instead observe the concept of nucleation and growth. A small portion becomes overdense with one type of particle. This then grows eventually forming the immiscible layers discussed above.}{binfluid}\end{mygraphic}
We can capture this phenomenology with a Landau-Ginzburg\index{free energy!Landau-Ginzburg} free energy, $F[\phi]$, where we now have the coarse grained variable;
\begin{equation}
\phi(\vec r) = \frac{\rho_A(\vec r) - \rho_B(\vec r)}{\tfrac{N_A}{V} + \tfrac{N_B}{V}}
\end{equation}
Then the free energy is;\footnotemark
\footnotetext{
Suppose we added a linear term as in magnetism, $\int{\upd{\vec r}\phi(\vec r)} = V\bar{\phi}$. But this is a constant for any given contents, so it has no effect.
}
\begin{equation}
\label{eq:symfe}
\beta F[\phi] = \int{\upd{\vec r}\underbrace{\tfrac{a}{2}\phi(\vec r)^2 + \tfrac{b}{4}\phi(\vec r)^4}_{f(\phi) \, \text{, local part}} + \tfrac{k}{2}\left(\nabla \phi\right)^2}
\end{equation}
Note that $a, b, k$ are all functions of $T$ (esp. $a$) and we can set $\beta = 1$ wlog.\footnote{It wouldn't be unreasonable to add a cubic term to the free energy to model for example the fact that the molecules may not be the same size, breaking the $\phi \mapsto -\phi$ symmetry. However we can always remove this by a linear shift in $\phi$ and $a$. So it's sufficient to just consider the symmetric free energy in \eqref{eq:symfe}.} We can do mean field theory (MFT)\index{mean field theory} for this model, looking for the single $\phi = \phi(\vec r)$ to minimise $F$ $\iff$ maximising $P \propto \exp -\beta F$. Now;
\begin{equation*}
\int{\upd{\vec r} \frac{k}{2}(\nabla \phi)^2} \geq 0
\end{equation*}
so we try a uniform $\phi(\vec r) = \bar{\phi}$. Then $F/V = f(\bar{\phi})$. This situatation is illustrated in \autoref{fig:mftbin}
\begin{mygraphic}{tpscm/mftbin}{0.6}{We keep $b$ fixed and vary $a = a(T)$. We can't change the global $\bar{\phi}$, but the mixture can split into two parts, this is the phenomenon of \emph{phase separation}\index{phase separation}.}{mftbin}\end{mygraphic}
So for $a < 0$ we have \emph{phase separation} with the \emph{binodal density}\index{density!binodal} given by;
\begin{equation}
\phi_{1,2} = \pm \phi_B = \sqrt{-\frac{a}{b}}
\end{equation}
This splitting is energetically favourable since for $-\phi_B < \bar{\phi} < \phi_B$;
\begin{equation}
V_1 f(\phi_1) + V_2 f(\phi_2) < (V_1 + V_2) f(\bar{\phi})
\end{equation}
This leads to the mean field phase diagram as shown in \autoref{fig:binfluid}. The \emph{spinodal instability}\index{spinodal instability} occurs when $f^{\prime \prime}(\phi) < 0 \Rightarrow -\phi_S < \bar{\phi} < \phi_S$, where the \emph{spinodal density}\index{density!spinodal} is defined by;
\begin{equation}
\phi_S = \sqrt{-\frac{a}{3b}}
\end{equation}
The gradient term $\int{\tfrac{k}{2}(\nabla \phi)^2 \upd{\vec r}}$ plays the role of capturing the $\mO(\nabla^{(2)})$ effect of non-local interactions. We can write the interaction energy, $E_{\text{\tiny{int}}}$ as;
\begin{equation}
E_{\text{\tiny{int}}} = \int{\ud \vec{r} \upd{\vec{r}\pr} \rho_i(\vec r) \rho_j(\vec{r}\pr)U_{ij}(\abs{\vec{r} - \vec{r}\pr})}
\end{equation}
If we Taylor expanded this, and assume that $\bar{\phi}$ is smooth on the scales of molecular attractions, we would find that $k$ fixes the \emph{interfacial tension}\index{interfacial tension}, $\sigma$, and the free energy of the phase separated state, where $A$ is the area of the interface\index{interface}.
\begin{equation}
\sigma = \left(-\frac{8ka^3}{9b^2}\right)^{\tfrac{1}{2}}, \qquad F = V_1 \phi_1 + V_2 \phi_2 + \sigma A
\end{equation}
\subsection{Landau-Ginzburg Theory for Nematic Liquid Crystals\index{nematic}\index{nematic!liquid crystal}} 
We have two phases to consider in the case of a \emph{nematic} liquid. The isotropic phase and the nematic phase, where the latter exhibits long range orientational order. As the density $\rho$ is increased, there is a discontinuous phase transition $I \mapsto N$. Nematic molecules have a preferred axis but no sense. We need an order parameter that identifies $\pm \vec n$. 
\begin{mygraphic}{tpscm/nematic}{0.7}{The nematic phase has a preferred alignment of molecules. Note that this is not externally induced, so it spontaneously breaks the rotational symmetry of the system.}{nematic}\end{mygraphic}
Suppose we consider an arbitrary vector $\vec A$, then for a molecule along $\hat{\vec{n}}$, we construct $(\vec A \cdot \hat{\vec n})\hat{\vec n}$. Then the second rank tensor $n_i n_j$ will act as the order parameter. In the isotropic case, it must be proportional to the rank $2$ isotropic tensor $\delta_{ij}$, so;
\begin{equation}
\left< n_i n_j \right> = \frac{\delta_{ij}}{d} \Rightarrow \tr(n_i n_j) = 1
\end{equation}
\begin{definitionbox}
Finally then we define the order parameter $Q_{ij}(\vec r)$;\footnote{Note here that the averaging requires an implicit coarse-graining description.}
\begin{equation}
Q_{ij}(\vec r) = \left< n_i n_j \right>_{\text{\tiny{local}}} - \frac{1}{d}\delta_{ij}
\end{equation}
\end{definitionbox}
So $Q_{ij}$ is a traceless symmetric rank $2$ tensor which vanishes in the isotropic phase. We need to generate a free energy functional that describes this phase transition. Suppose we have a local part $f(Q)$, then we should built this up as a Taylor series in the scalar invariants of $Q$. In increasing powers of $\text{Q}$ these are given by;
\begin{center}
\begin{mytable}{lc}
\textbf{Order} 	& \textbf{Scalar Invariant}						\\ \midrule
Linear 		& $Q_{ii} = \tr Q = 0$							\\
Quadratic 		& $Q_{ij}Q_{ji} = \tr \left(Q^2\right)$				\\
Cubic 		& $Q_{ij}Q_{jk}Q_{ki} = \tr\left(Q^3\right)$			\\
Quartic 		& $\tr\left(Q^4\right)$, $\left(\tr\left(Q^2\right)\right)^2$		
\end{mytable}
\captionof{table}{The scalar invariants of the rank $2$ tensor $Q$ in increasing powers of $Q$.}
\end{center}
So we can write down a free energy;\footnote{Unlike in the binary fluid case, we must have $c \neq 0$ here. Indeed the binary fluid case of a conserved scalar is the only case in which the cubic term can be removed rather than the norm.}
\begin{equation}
f(Q) = a \tr(Q^2) + c \tr(Q^3) + b_1 \left(\tr(Q^2)\right)^2 + b_2 \tr(Q^4)
\end{equation}
Consider two different states, $Q^n_{ij}$ with $\lambda > 0$;\footnotemark
\footnotetext{
Note that these are in fact the \emph{only} cases we need to consider. Since $Q$ is symmetric, we can always diagonalise it with an orthogonal transformation. In other words we can always align our axes such that $Q$ is diagonal. Then there are only two possible signatures for traceless matrices in $3$-dimensions; $(+, -, -)$ and $(-, +, +)$. The fact that the two eigenvalues with the same sign have been set to be equal is not inherently general, but it does not change the physics qualitatively, just involving a redefinition of the free energy coefficients. 
}
\begin{equation*}
Q^1_{ij} = \thrbythr{-\tfrac{\lambda}{2} & 0 & 0}{0 & -\tfrac{\lambda}{2} & 0}{0 & 0 & \lambda}, \quad Q^2_{ij} = \thrbythr{\tfrac{\lambda}{2} & 0 & 0}{0 & \tfrac{\lambda}{2} & 0}{0 & 0 & -\lambda}
\end{equation*}
\begin{mygraphic}{tpscm/probdistr}{0.5}{The two states $Q^{1,2}_{ij}$ give rise to differing probability distributions for the orientation of the vector. In the isotropic case (blue), the molecules point any way in a plane and there is no long range order, whilst in the nematic case (red), there is an alignment along the $x^3$ axis.}{probdistr}\end{mygraphic}
We refer to the case $Q^1$ as the \emph{uniaxial}\index{uniaxial} state. This leads to;
\begin{align*}
f(Q^1) = f(\lambda) &= a(\tfrac{3}{2}\lambda^2) + c(\tfrac{3}{4}\lambda^3) + b_1(\tfrac{9}{4}\lambda^4) + b_2(\tfrac{9}{8}\lambda^4) \\
&= \bar{a}\lambda^2 + \bar{c}\lambda^3 + \bar{b} \lambda^4
\end{align*}
Now we fix $\bar{b}, \bar{c}$ and vary $\bar{a} = \bar{a}(T)$ as before. Now the cubic term gives a discontinuous phase transition once $\bar{a} = \bar{a}_c$; the global minimum jumps from $\lambda = 0$ to some finite positive $\lambda_0$.
\begin{mygraphic}{tpscm/nematicpt}{0.5}{The minimum at $\lambda = \lambda_c$ moves below zero at a critical value of $a$ which determines the point at which the discontinuous phase transition occurs.}{nematicpt}\end{mygraphic}
What about the gradient terms? We want to generate a scalar contribution to $F[Q]$. At order $\nabla^{(2)}$ we can write;
\begin{align*}
f_{\nabla} &= k_1 \nabla_i \nabla_i Q_{jl}Q_{jl} + k_2 \left(\nabla_i Q_{im}\right) \left(\nabla_j Q_{jm}\right) + k_3 \left(\nabla_i Q_{jm}\right)\left(\nabla_j Q_{im}\right) \\
&= k_1\nabla^2 \tr(Q^2) + k_2 \left(\nabla \cdot Q\right)^2 + k_3\left(\nabla_i Q_{jm}\right)\left(\nabla_j Q_{im}\right) 
\end{align*}
We usually set $k_1 = k_3 = 0$ and work only with the middle term.\footnote{This is usually justified by the fact that the three modes give a similar energy cost and as such, the dynamics associated to the gradient term can be described sufficiently with just the single term.}
\subsection{Functional Integration}
We consider a scalar functional;
\begin{equation}
A[\phi] = \int{\upd{\vec r} \mL(\phi, \nabla\phi)}
\end{equation}
and a variation $\phi(\vec r)\mapsto \phi(\vec r) + \delta \phi(\vec r)$ such that $\delta\phi = 0$ on the boundary. Then varying $A[\phi]$ gives the standard functional derivative\index{functional!derivative};
\begin{equation}
\frac{\delta A}{\delta \phi(\vec r)} = \frac{\del \mL}{\del \phi(\vec r)} - \nabla \cdot \frac{\del \mL}{\del \nabla \phi}
\end{equation}
Mean field theory sets this variation equal to zero; $\delta A/\delta \phi = 0$, which produces a single solution that extremises the free energy. e.g. the domain wall/interface profile $\phi(x) = \phi_B \tanh(x-x_0/\xi_0)$. In a general sense we can think of $\delta F / \delta \psi(\vec r)$ as a `force' that tells the system which was to move to minimise the free energy. In analogy, consider the thermondynamic relation $\ud F = -S\ud T - p\ud V + \mu \ud N + \cdots$, then we have;
\begin{equation}
\delta F = \int{\upd{\vec r} \underbrace{\frac{\delta F}{\delta\psi(\vec r)}}_{\mu \text{ like}} \underbrace{\delta \psi(\vec r)}_{N \text{ like}}}
\end{equation}
If $\psi$ is a conserved scalar, the $\mu[\phi] = \delta F/\delta\psi(\vec r)$ is indeed called the \emph{chemical potential}\index{chemical potential}. If $\psi$ is not conserved, then we have a more general concept called the \emph{molecular field}\index{molecular field};\footnotemark
\begin{equation}
H_{ij\ldots k} \coloneqq \frac{\delta F}{\delta Q_{ij \ldots k}(\vec r)}
\end{equation}
\footnotetext{
This is significant for hydrodynamic\index{hydrodynamic} equations; suppose we have a conserved variable $\dot{\phi} = - \nabla \cdot \vec J$ where $J \propto -\nabla \mu(\vec r)$. Then the system moves from high to low concentrations as determined by $\mu$. In the non-conserved case, we have a similar idea known as \emph{local relaxation}\index{local relaxation}; $\dot{Q}_{ij\ldots k} = -\Gamma H_{ij\ldots k}$.
}
\subsubsection{Stress Tensor for scalar $\phi(\vec r)$}
We consider a small displacement; $\vec r \rightarrow \vec r + \vec{u}(\vec{r})$ where for simplicity we assume that $\vec{u}$ is incompressible i.e. $\nabla\cdot \vec{u}= 0$. Then let $\phi\mapsto \phi\pr$ such that $\phi\pr\left(\vec r) = \phi(\vec r - \vec{u}(\vec r)\right)$, so;\footnote{Note that this is just a passive transformation\index{transformation!passive}}
\begin{equation}
\delta \phi(\vec r) = \phi\pr(\vec r) - \phi(\vec r) = - \vec{u} \cdot \nabla \phi + \mO(u^2)
\end{equation}
Then $\delta F$ is given by;
\begin{align*}
\delta F = \int{\upd{\vec r}\delta \phi \frac{\delta F}{\delta \phi(\vec r)}} &= - \int{\upd{\vec r}\mu(\vec r)\vec{u}\cdot \nabla \phi} \\
&= \int{\upd{\vec r}\phi \nabla\cdot(\mu \vec{u})} \\
&= \int{\upd{\vec{r}}(\phi \nabla \mu)\cdot \vec{u}} + \int{\upd{\vec r}\phi\mu \nabla\cdot \vec{u}} \\
&= \int{\upd{\vec r} (\phi \nabla \mu)\cdot \vec{u}} \\
&= \int{\upd{\vec r}\phi(\nabla_j \mu)u_j}
\end{align*}
But we could also do the calculation from the other perspective; in terms of the work done by the stress\index{stress}, $\sigma_{ij}(\vec r)$, and the strain\index{strain}; $\epsilon_{ij} = \nabla_i u_j$.
\begin{align*}
\delta F &= \int{\upd{\vec r} \sigma_{ij}(\vec r) \epsilon_{ij}} \\
&= \int{\upd{\vec r}\sigma_{ij}\nabla_i u_j} = -\int{\upd{\vec r}(\nabla_i \sigma_{ij})u_j}
\end{align*}
Comparing the two expressions we see that $\mu$ contains mechanical information also via;
\begin{equation}
\nabla_i \sigma_{ij} = -\phi \nabla_j \mu, \qquad \nabla \cdot \sigma = -\phi \nabla\mu
\end{equation}
\subsection{Functional Integrals}
Suppose we have a coarse-grained\index{coarse-graining} description of a system described by $\psi(\vec r)$, and a free energy functional $F[\psi] = E[\psi] - TS[\psi]$, then;
\begin{equation}
\exp(-\beta F_{\text{tot}}) = \mZ_{\text{tot}} = \int{\mD[\psi]\,\,e^{-\beta F[\psi]}}
\end{equation}
We will consider taking $F[\psi] = F\set{\psi_{\vec{q}}}$, which for a finite system is a function of a finite set of $\set{\vec{q}}$ and;
\begin{equation}
\psi_{\vec{q}} = \frac{1}{\sqrt{V}}\int{\upd{\vec r}\psi(\vec r)e^{-i\vec{q}\cdot\vec{r}}}
\end{equation}
where the volume factor ensures $\int{\upd{\vec r}\abs{\psi(\vec{r})}^2} = \sum_{\vec{q}}{\abs{\psi_{\vec{q}}}^2}$\index{theorem!Parseval}. Then we have a countable set of real variables $\set{\psi_{\vec{q}}}$ from which we define the measure;
\begin{equation}
\mD[\psi] \coloneqq \prod_{\vec{q}}{\ud \psi_{\vec{q}}}
\end{equation}
By coarse-graining the molecular physics, there is a scale $\vec{q}_{\text{max}}$ which sets the minimum scale of variations. This has the effect of nullifying the divergence of integrals at large $\vec{q}$ for example. In practice there is only one integral we can actually do like this where $\beta F$ is a quadratic form;\footnotemark
\footnotetext{
We will do this explicitly in this case to illustrate the steps, we use (modulo factors of the volume);
\begin{equation*}
\phi(\vec{r}) = \sum_{\vec{q}}{\phi_{\vec{q}}e^{i\vec{q}\cdot\vec{r}}}, \qquad \int{\upd{\vec{r}}e^{i\vec{q}\cdot\vec{r}}} = \delta(\vec{q})
\end{equation*}
Then the first term in the quadratic form becomes;
\begin{equation*}
\tfrac{1}{2}\int{\sum_{\vec{q}_1, \vec{q}_2, \vec{q}_3}{\ud{\vec{r}}\upd{\vec{r}\pr}\phi_{\vec{q}_1}\phi_{\vec{q}_2}e^{i\vec{r}\cdot(\vec{q}_1 + \vec{q}_3) + i\vec{r}\pr\cdot(\vec{q}_2 - \vec{q}_3)}H_{\vec{q}_3}}}
\end{equation*}
which gives the result $\tfrac{1}{2}\sum_{\vec{q}}{\phi_{\vec{q}}\phi_{-\vec{q}}H_{\vec{q}}}$ on integrating over $\vec{r}, \vec{r}\pr$.
}
\begin{align}
\beta F[\phi] &= \tfrac{1}{2}\int{\ud\vec{r}\upd{\vec{r}\pr}\phi(\vec{r})G(\vec{r} - \vec{r}\pr)\phi(\vec{r}\pr)} - \int{\upd{\vec{r}}h(\vec{r})\phi(\vec{r})} \nonumber \\
\Rightarrow \beta F[\phi_{\vec{q}}] &= \tfrac{1}{2}\sum_{\vec{q}}{G(\vec{q})\phi_{\vec{q}}\phi_{-\vec{q}}} - \sum_{\vec{q}}{h_{\vec{q}}\phi_{\vec{q}}}
\end{align}
We apply this form in the Gaussian model;
\begin{align*}
\beta F[\phi] &= \int{\upd{\vec{r}}\set{\tfrac{a}{2}\phi^2 + \left.\tfrac{b}{4}\phi^4\right|_{b = 0} - h\phi + \tfrac{k}{2}(\nabla\phi)^2 + \tfrac{\gamma}{2}(\nabla^2 \phi)^2 + \cdots}} \\
\Rightarrow \beta F[\phi_{\vec{q}}] &= \tfrac{1}{2}\sum_{\vec{q}}{(a + kq^2 + \gamma q^4)\phi_{\vec{q}}\phi_{-\vec{q}}} - \sum_{\vec{q}}{h_{\vec{q}}\phi_{\vec{q}}}
\end{align*}
Then the partition function is given by;
\begin{equation*}
\mZ_{\text{tot}} = \int{\left(\prod_{\vec{q}}^{+}{\ud \phi_{\vec{q}}}\right)}
\end{equation*}
where we have restricted the product to a half-space (say $\vec{q}_{x_1} > 0$). This is not a trivial statement and there is an extended discussion of this point in the Statistical Field Theory course. Ultimately it arises from the fact that $\phi_{\vec{q}}$ and $\phi_{\vec{q}}^{\star}$ are not truly independent in the case of a real scalar field $\phi(\vec{r})$. Instead we have $\phi^{\star}_{\vec{q}} = \phi_{-\vec{q}}$, this implies;
\begin{equation*}
\phi_{\vec{q}}\phi_{-\vec{q}} = \left(\text{Re}(\phi_{\vec{q}})^2 + \text{Im}(\phi_{\vec{q}})^2\right) = \abs{\phi_{\vec{q}}}^2
\end{equation*}
Then we have ($h = 0$);
\begin{align*}
\mZ_{\text{tot}} &= \int{\left[\prod_{\vec{q}}^{+}{\ud \phi_{q}}\right]\exp\left(-\sum_{\vec{q}}^{+}{\phi_{\vec{q}}\phi_{-\vec{q}}G(q)}\right)} \\
&= \prod_{\vec{q}}^{+}{\int{\upd{\phi_{\vec{q}}}\left(\exp-\abs{\phi_{\vec{q}}}^2 G(q)\right)}}
\end{align*}
If we focus on only one of these integrals, let $\phi_{\vec{q}} = \rho e^{i\theta}$, then $\ud \phi_{\vec{q}} = \rho\ud \rho \ud \theta$, then;
\begin{align*}
\int{\upd{\phi_{\vec{q}}} \exp\left(-\abs{\phi_{\vec{q}}}^2G(q)\right)} &= \int{\rho \ud \rho \upd{\theta} \exp\left(-G(q)\rho^2\right)} \\
&= 2\pi \int_{0}^{\infty}{\rho \upd{\rho}e^{-G(q)\rho^2}} = \frac{\pi}{G(q)} \\
\Rightarrow \mZ_{\text{tot}} &= \prod_{\vec{q}}^{+}{\frac{\pi}{G(q)}}
\end{align*}
Then we finally have that;
\begin{equation}
\beta F_{\text{tot}} = -\log \mZ_{\text{tot}} = \sum_{\vec{q}}^{+}{\log\left(\frac{G(q)}{\pi}\right)}
\end{equation}
\begin{examplebox}[Correlation Functions]
This allows to compute the correlation function $S(\vec{k}) = \left< \phi_{\vec{k}} \phi_{-\vec{k}} \right>$;
\begin{align*}
S(\vec{k}) &= \frac{1}{\mZ_{\text{tot}}} \int{\left[\prod_{\vec{q}}^{+}{\ud \phi_{q}}\right]\phi_{\vec{k}}\phi_{-\vec{k}}\exp\left(-\sum_{\vec{q}}^{+}{\phi_{\vec{q}}\phi_{-\vec{q}}G(q)}\right)} \\
&= \frac{1}{\mZ_{\text{tot}}}\frac{\delta \mZ_{\text{tot}}}{\delta G(\vec{k})} = -\frac{\delta \log \mZ_{\text{tot}}}{\delta G(\vec{k})} \\
&= \frac{\delta \beta F_{\text{tot}}}{\delta G(\vec{k})} = \frac{1}{G(k)}
\end{align*}
\end{examplebox}
We now want to ask the question as to what happens when we put back in the quartic term. We know that the Gaussian fluctuations do not describe the region surrounding the critical point well. This leads to the renormalisation group\index{renormalisation group}. Between this mean field theory\index{mean field theory} approach and RG, there is a variational method known as \emph{Hartree Theory}\index{Hartree theory} in condensed matter physics, and \emph{1-loop self constency} in QFT.
\subsection{Hartree Theory}\label{sec:hartree}
Without loss of generality we set $\beta = 1$, then we change notation $F_{\text{tot}} \mapsto F, F[\phi] \mapsto H[\phi]$. $H[\phi]$ is now known as the effective Hamiltonian\index{effective Hamiltonian}\index{Hamiltonian!effective}. Suppose we introduce a trial Hamiltonian, $H_{0}$, then the trial free energy, $F_{0}$,satisfies
\begin{equation*}
e^{-F_{0}} = \int{\mD\phi\,\,e^{-H_{0}[\phi]}}
\end{equation*}
Hence by a trivial insertion we have;
\begin{align*}
e^{-F} &= \frac{e^{-F_{0}}}{\int{\mD\phi\,\,e^{-H_{0}}}} \int{\mD\phi\,\,e^{-H_{0}}e^{-(H - H_0)}} \\
&= e^{-F_0} \left< e^{-(H - H_0)} \right>_0 \\
\Rightarrow F &= F_0 - \log\left< e^{-(H - H_0)} \right>_0
\end{align*}
If we think about the $\log$ graph we will easily deduce that;
\begin{align*}
\log(aA + bB) &\geq a\log A + b\log B \\
\Rightarrow \log\left< Y \right>_0 &\geq \left< \log Y \right>_0
\end{align*}
Applying this to the free energy expression above, we find that;
\begin{equation}
F \leq F_{0} - \left< H_0 \right>_0 + \left< H \right>_0
\end{equation}
which is known as the \emph{Feynman-Bogolinbov inequality}\index{Feynman-Bogolinbov inequality} and is very similar in character to the variational method\index{variational method} in quantum mechanics.
\subsection{Isotropic to Smectic\index{smectic} Transition}
The type of materials this transition describes includes dimers\index{dimer}, hydrophilic molecules and copolymers\index{copolymer} where there are two distinct groups that have differing properties. For simplicity we deal with the symmetric case first. The hallmark of a smectic transition is translational/orientational order in one direction and liquid order in the other two. This known as a \emph{smectic crystal}\index{smectic crystal}. Before we write down the free energy, it is useful to discuss the order parameter. As we move to more complicated systems, the simple interpretation of the order parameter\index{order parameter} as the quantity that vanishes in the disordered phase and is non-zero in the ordered phase has to be modified. Here, we have a functional dependence that characterises the \emph{periodic order} of the smectic phase. Then we understand the order parameter contains an amplitude that determines how strongly this periodic order in some preferred direction is manifest in the system. If the order is present and there many chains $ABBAABBAAB\cdots$ in one direction, then we say we are in the smectic phase. Now, the Landau-Ginzburg\index{Landau-Ginzburg theory} is;
\begin{align*}
\beta F[\phi] &= \int{\upd{\vec{r}}\set{\tfrac{a}{2}\phi^2 + \tfrac{b}{4}\phi^4 + \tfrac{k}{2}(\nabla\phi)^2 + \tfrac{\gamma}{2}(\nabla^2\phi)^2}} \\
&= \tfrac{1}{2}\sum_{\vec{q}}{(a + kq^2 + \gamma q^4)\phi_{\vec{q}}\phi_{-\vec{q}}} + \tfrac{b}{4}\sum_{\vec{q}_1, \vec{q}_2, \vec{q}_3}{\phi_{\vec{q}_1}\phi_{\vec{q}_2}\phi_{\vec{q}_3}\phi_{-(\vec{q}_1 + \vec{q}_2 + \vec{q}_3)}}
\end{align*}
For the isotropic-smectic transition we want to take $k < 0, \gamma > 0$. This ensures two things; firstly the free energy is still bounded for rapidly oscillating field profiles, supressed by $\gamma$. Secondly, it means there is a positive solution $q_0 > 0$ to $G(q) = a + kq^2 + \gamma q^4 = 0$ before $a < 0$. Now the idea is we start by working in the Gaussian regime. Here we will find a preferred wavenumber $q_0$ that dominates over any fluctuations in the absence of the quartic term. We will ultimately find that adding in the quartic term is equivalent to this dominant mode being suppressed, ensuring that the smectic order is not seen until $a < a_c$ as defined below, characterising the fact that it is in fact a first order transition.
\begin{mygraphic}{tpscm/strucfact}{0.9}{For $a > 0$ there is a solution $G(q_0) = 0$ with $q_0 \neq 0$. This leads to a divergence in the structure factor $S(q)$. This governs the preferred length scale in the system, $q_0 = 2\pi/\lambda$.}{strucfact}\end{mygraphic}
As $a \rightarrow 0$, we can find the first instability by setting $G(q_0) = 0 = G\pr(q_0)$. We find that;
\begin{equation*}
q_0 = \sqrt{-\frac{k}{2\gamma}}, \qquad a_c = \frac{k^2}{4\gamma}
\end{equation*}
We use these to re-expand $G$ to second order in $(q - q_0)^2$;
\begin{equation*}
G(q) = \tau + \alpha (q - q_0)^2, \quad \tau = a - a_c, \quad \alpha = -2k > 0
\end{equation*}
If we now take a mean field theory approach with $\phi = A \cos q_0 z$, which minimises the free energy, the magnitude of $A$ will act as our order parameter, with the smectic phase emerging when $A \neq 0$. So that the explicit free energy is well-defined, we take the average over one period to find that;
\begin{align*}
\frac{\beta F}{V} &= \frac{1}{4}\left(a A^2 + kA^2 q_0^2 + \gamma A^2 q_0^4  + \frac{3b}{8}A^4\right) \\
&= \frac{1}{4}\left(A^2 (a - a_c) + \frac{3b}{8}A^4\right) = \frac{1}{4}\left(\tau A^2 + \frac{3b}{8}A^4\right)
\end{align*}
We have seen this functional form multiple times, and recognise it as describing a continuous transition to the layered state $A \neq 0$ when $\tau < 0 \iff a < a_c$. This is shown in \autoref{fig:Aa};
\begin{mygraphic}{tpscm/Aa}{0.4}{The phase diagram showing the periodic component of $\phi$ turning on for low values of $a < a_c$. This predicts a second order transition; we will in fact find it is first order, the fluctuations destroying the smectic order for low values of $A$.}{Aa}\end{mygraphic}
\subsubsection{Applying the Variational Method}
We now want to include the effect of fluctuations. This will ultimately result in the phase transition becoming first order. We follow the procedure in Section \ref{sec:hartree} with;
\begin{equation}
H = \sum_{\vec{q}}^{+}{\phi_{\vec{q}}\phi_{-\vec{q}}G(q)} + \frac{b}{4V}\sum_{\set{\vec{q}_i}}{\phi_{\vec{q}_1}\phi_{\vec{q}_2}\phi_{\vec{q}_3}\phi_{-(\vec{q}_1 + \vec{q}_2 + \vec{q}_3)}}
\end{equation}
where $G(q) = \tau + \alpha(q-q_0)^2$. We now choose a trial Hamiltonian to be Gaussian with an arbitrary kernel $J(q)$;
\begin{equation}
H_0 = \sum_{\vec{q}}^{+}{\phi_{\vec{q}}\phi_{-\vec{q}}J(q)} \Rightarrow F_0 = \sum_{\vec{q}}^{+}{\log\left(\frac{J(q)}{\pi}\right)}
\end{equation}
We will make use of the fact that $\left< \phi_{\vec{q}} \phi_{-\vec{q}} \right> = J(q)^{-1}$ so that;
\begin{align*}
\left< H_0 \right>_0 &= \sum_{\vec{q}}^{+}{\left< \phi_{\vec{q}}\phi_{-\vec{q}} \right>J(q)} \\
&= \sum_{\vec{q}}^{+}{\frac{1}{J(q)}J(q)} = \sum_{\vec{q}}^{+}{1} \\
\left< H \right>_0 &= \sum_{\vec{q}}^{+}{\frac{1}{J(q)} G(q)} + \frac{b}{4V}\sum_{\set{\vec{q}_i}}{\left< \phi_{\vec{q}_1}\phi_{\vec{q}_2}\phi_{\vec{q}_3}\phi_{-(\vec{q}_1 + \vec{q}_2 + \vec{q}_3)} \right>_0}
\end{align*}
We can use Wick's theorem to calculate the second expectation in $\left< H \right>_0$, let $\left< \phi_{\vec{q}_1}\phi_{\vec{q}_2}\phi_{\vec{q}_3}\phi_{-(\vec{q}_1 + \vec{q}_2 + \vec{q}_3)} \right>_0 \coloneqq U$, then;
\begin{align*}
U &= 3\set{\sum_{\vec{q}}{\frac{1}{J(q)}}}^2 = 12 \set{\sum_{\vec{q}}^{+}{\frac{1}{J(q)}}}^2 \\
\Rightarrow \left< H \right>_0 &= \sum_{\vec{q}}^{+}{\frac{G(q)}{J(q)}} + \frac{3b}{V} \set{\sum_{\vec{q}}^{+}{\frac{1}{J(q)}}}^2
\end{align*}
Then our estimate of the free energy $F$ is;
\begin{equation}
\sum_{\vec{q}}^{+}{\set{\log\left(\frac{J(q)}{\pi}\right) - 1 + \frac{G(q)}{J(q)}}} + \frac{3b}{V} \set{\sum_{\vec{q}}^{+}{\frac{1}{J(q)}}}^2
\end{equation}
We want to minimise this over all the possible $J(q)$ which gives;
\begin{equation*}
\frac{1}{J(q)} - \frac{G(q)}{J(q)^2} - \frac{6b}{V J(q)^2}\sum_{\vec{q}\pr}^{+}{\frac{1}{J(q\pr)}} = 0
\end{equation*}
This further implies that;
\begin{equation*}
J(q) = G(q) + \frac{6}{V}\sum_{\vec{q}\pr}^{+}{\frac{1}{J(q\pr)}}
\end{equation*}
Taking the large $V$ limit we have;\footnotemark
\footnotetext{
This is the mapping;
\begin{equation*}
\frac{(2\pi)^d}{V} \mapsto \ud^d q \Rightarrow \frac{1}{V}\sum_{\vec{q}}^{+} \mapsto \frac{1}{2(2\pi)^d}\int{\ud^d q}
\end{equation*}
}
\begin{equation}
J(q) = G(q) + \frac{3b}{(2\pi)^d}\int{\frac{\ud q\pr}{J(q\pr)}}
\end{equation}
We want to apply this in the case of the isotropic-smectic transition. Since the integral part of this expression, for $J(q)$ to be consistent with $G(q) = \tau + \alpha(q - q_0)^2$, it must be the case that $J(q) = \bar{\tau} + \alpha(q - q_0)^2$. Then;
\begin{equation*}
\bar{\tau} = \tau + \frac{3b}{(2\pi)^d}\int{\frac{\ud^d q\pr}{\bar{\tau} + \alpha(q-q_0)^2}}
\end{equation*}
If we are close to a critical point, then the integral is dominated by the contribution at $q = q_0$, so for $d = 3$;
\begin{equation*}
\frac{3b}{(2\pi)^3}\int{\frac{\ud^3 q}{\bar{\tau} + \alpha(q-q_0)^2}} \mapsto q_0^2 \frac{3b}{2\pi^2}\int_0^{q_{\text{max}}}{\frac{\ud q}{\bar{\tau} + \alpha(q - q_0)^2}}
\end{equation*}
With the substitution $\sqrt{\alpha}(q - q_0) = \sqrt{\tau}\tanh y$ we find that;
\begin{equation}
\bar{\tau} = \tau + \frac{sb}{\sqrt{\bar{\tau}}}, \qquad s = \frac{3q_0^2}{2\pi^2 \sqrt{\alpha}}\int{\frac{\ud \xi}{1 + \xi^2}}
\end{equation}
This is a self-consistency that ensures $\tau$ can \emph{never be zero}. So we cannot have $J(q)$ infinitely peaked. Thus we cannot have the continuous transformation as we saw before. For small $\tau$ we still have large fluctuations at some preferred wavenumber, but these are self-limiting. The other modes are still contributing and their random fluctuations overwhelm those of the preferred frequency. This is different to the mean field case where the fluctuations couldn't overcome the `infinitely preferred' $q_0$. 

\paraskip
So, the isotropic-smectic transition is not continuous, and is known as a \emph{fluctuation-induced}\index{transition!fluctuation-induced} or \emph{Brazovskii}\index{transition!Brazovskii} transition. To examine the details further, we consider an ordered state $\phi = \phi_0 + \delta \phi$, where $\phi_0 = A \cos q_0 z$. This really comes from a new trial Hamiltonian;
\begin{equation*}
H_0 = \sum_{\vec{q}}^{+}{\phi_{\vec{q}}\phi_{-\vec{q}}J(q)} - h A
\end{equation*}
where $h$ is a Lagrange multiplier that we minimise over. This gives a new estimate of the free energy, $\tilde{F}(A)$. When $A = 0$ this just coincides with the discussion above, for $A \neq 0$ we will find;
\begin{equation}
\bar{\tau} = \tau + \frac{sb}{\sqrt{\bar{\tau}}} + \frac{3bA^2}{2}
\end{equation}
Here, the first two terms come from the fluctuations, $\delta\phi$, whilst the last term arises due to $\phi_0$. At small values of $A$, the fluctuations dominate and drive the first order character of the transition. On the other hand at large $A$, the fluctuations vanish and MFT is a good estimate. In general $\tilde{F}(A)$ will be a complicated function of $A$ for each $\tau$. Nonetheless, we can qualitatively sketch the situation, noting that near $A = 0$, it must be the case that $\bar{\tau} > 0$. Then the point $A = 0$ must always be a minimum of the free energy. This generates the following plot; 
\begin{mygraphic}{tpscm/smec}{0.7}{The estimate of the free energy using the variational method, $\tilde{F}(A)$, is shown in red. There is a critical $\tau = \tau_c$ where the minimum at $A = A_c$ moves below the axis and becomes the global minimum. It also illustrates the fact that at large $A$, the blue curves i.e. MFT are still a good approximation.}{smec}\end{mygraphic}
So minimising the free energy $\tilde{F}(A)$ over $A$ now produces a discontinuous phase diagram;
\begin{mygraphic}{tpscm/mfthart}{0.5}{The phase diagram illustrating the difference between the mean field theory calculation and the more precise fluctuation-induced transition predicted by Hartree theory}{<label>}\end{mygraphic}
The details of this require the leading order behaviour of $\tilde{F}(A)$ and can be found in \href{http://www.jetp.ac.ru/cgi-bin/dn/e_041_01_0085.pdf}{Brazovskii, 1975}.
\subsubsection*{Brazovskii Transition with a cubic term $c\phi^3$}
This would give an $A^3$ term at MFT level and leads to a discontinuous transition. As it happens, $c$ only matters above a threshold value, below which we see the same isotropic-smectic transition as above. Above this critical point, there is a new hexagonal phase that exhibits two dimensional crystalline structure along with one-dimensional liquid order in the other direction.
\newpage
\section{Dynamics}
\subsection{Building a Toolbox}
We start by simply considering the classical mechanics of a $1$D particle with $A = \int{\upd{t}L}$, $L = T - V$. Then we have the variation;
\begin{equation*}
\frac{\delta A}{\delta x(t)} = \frac{\del L}{\del x} - \frac{\ud}{\ud t}\left(\frac{\del L}{\del \dot{x}}\right)
\end{equation*}
We make two notes about this;
\begin{enumerate}
\item It is deterministic
\item We will assume time reversal symmetry in $L$ reflecting the time reversal symmetry of the underlying microscopic laws. Furthermore, we will enforce that $L$ is explicitly independent of time. This is rationalised by the fact that we want the system to have a possible equilibrium state. In the presence of forcing, this cannot be the case.
\end{enumerate}
Now immerse the particle in a fluid bath (a colloid\index{colloid}) which introduces new terms;
\begin{enumerate}
\item Damping, $F_D = -\xi \dot{x}$
\item Noise force, $f$ such that $\left< f \right> = 0$
\end{enumerate}
Then the \emph{Langevin equation}\index{equation!Langevin} that governs the stochastic evolution of the system is;
\begin{equation}
-\frac{\delta A}{\delta x} = - \xi \dot{x} + f
\end{equation}
The noise term is the sum of many independent contributions so we expect it to be Gaussian (by some invocation of the central limit theorem\index{theorem!central limit}). Furthermore we assume that $f(t)$ is a Gaussian field in the same sense as page \pageref{sec:grf}. Then the \emph{white noise}\index{white noise} assumption involves identifying that $\left< f(t) f(t\pr) \right> = \sigma^2 \delta(t - t\pr)$ so that the covariance matrix is diagonal and isotropic. Then;
\begin{equation}
\mathbb{P}\left[f(t)\right] = \mathcal{N}_f \exp\left(-\frac{1}{2\sigma^2}\int{\upd{t}f(t)^2}\right)
\end{equation} 
where $\mathcal{N}_f$ is a normalisation constant. We really want the path probability measure for a trajectory $x(t)$ between $x_1$ at $t_1$ and $x_2$ at $t_2$. We have;
\begin{equation*}
f = \xi \dot{x} - \frac{\delta A}{\delta x}
\end{equation*}
from the Langevin equation, so the forward probability of a path $x(t)$ is;
\begin{equation}
\mathbb{P}_F[x(t)] = \mathcal{N}_x\exp\left(-\frac{1}{2\sigma^2}\int_{t_1}^{t_2}{\upd{t}\abs{\xi \dot{x} - \frac{\delta A}{\delta x}}^2}\right)
\end{equation}
Now instead consider the backwards path $x(t_2 - t)$ where we start at $x_2$ and end at $x_1$. On time reversal, $-\delta A/\delta x$ is unchanged by assumption, but $\xi \dot{x} \mapsto - \xi \dot{x}$ and $\ud t \mapsto -\ud t$. Then we find;
\begin{equation}
\mathbb{P}_B[x(t)] = \mathcal{N}_x \exp\left(-\frac{1}{2\sigma^2}\int_{t_1}^{t_2}{\upd{t}\abs{-\xi \dot{x} - \frac{\delta A}{\delta x}}^2}\right)
\end{equation}
Expanding the absolute values, we then calculate;\footnotemark
\footnotetext{
In going from the first to the second line we have defined the Hamiltonian function;
\begin{equation*}
H(x, \dot{x}) = \dot{x}\frac{\del L}{\del \dot{x}} - L
\end{equation*}
We then find that;
\begin{align*}
\frac{\ud H}{\ud t} &= \ddot{x}\frac{\del L}{\del \dot{x}} + \dot{x}\frac{\ud}{\ud t}\frac{\del L}{\del \dot{x}} - \left(\dot{x}\frac{\del L}{\del x} + \ddot{x}\frac{\del L}{\del \dot{x}}\right) \\
&= -\dot{x}\frac{\delta A}{\delta x}
\end{align*}
So integrating we just find the difference in the Hamiltonian function at the end points.
}
\begin{align*}
\frac{\mathbb{P}_F}{\mathbb{P}_B} &= \exp\left(-\frac{1}{2\sigma^2}\int_{t_1}^{t_2}{\upd{t}4\xi \left(-\dot{x}\frac{\delta A}{\delta x}\right)}\right) \\
&= \exp\left(-\frac{4\xi}{2\sigma^2}\left(H(x_2, \dot{x}_2) - H(x_1, \dot{x}_1)\right)\right)
\end{align*}
\subsubsection*{The Principle of Detailed Balance\index{detailed balance}}
Ultimately we want to fix the noise variance to keep the time reversal symmetry manifest in the system. Heuristically, the dissipations tend to drive the system to the ground state energy and break the time reversal invariance. We should thus introduce random fluctuations that counteract this decay. To do so we make a link we time invariant stochastic systems. Suppose we have a set of microstates, $\set{i}$ with probabilities $p_{i}$. Then, let the probability for transition between microstates $i$ and $j$ be $t_{ij}$. The principle of time reversal symmetry states that the rate of the process $i \rightarrow j$ should be equal to the rate for $j \rightarrow i$. As such it must be the case that;
\begin{equation}
p_i t_{ij} = p_j t_{ji}
\end{equation}
These are the detailed balance equations. In particular suppose we have a statistical system with microstates obeying a Boltzmann-like distribution (or indeed we can generalise this to the mesoscopic\index{mesostate} case by considering the effective free energy for the bulk collection of microstates). Then the principle of detailed balance tells us;
\begin{equation*}
e^{-\beta H_{i}}\mathbb{P}(i \rightarrow j) = e^{-\beta H_{j}} \mathbb{P}(j \rightarrow i)
\end{equation*}
The guiding principle in this statistical system is that if this were not the case then there would be no chance of an equilibrium in the system. It is a fundamental consequence of microscopic reversibility. We can now apply this to the situation under consideration; working with the Hamiltonian function above as an effective free energy, the following relation should hold;
\begin{equation}
\label{eq:detbal}
e^{-\beta H_1} \mathbb{P}_F = e^{-\beta H_2}\mathbb{P}_B
\end{equation} 
which can be read as the probability of starting at $x_1(t_1)$ and following a trajectory that finishes at $x_2(t_2)$ should be equal to the probability that we start at $x_2(t_2)$ and following the reverse trajectory.
\begin{definitionbox}
Matching \eqref{eq:detbal} up with the expression for $\mathbb{P}_F/\mathbb{P}_B$ derived above we deduce that we must have;
\begin{equation}
\frac{4\xi}{2\sigma^2} = \beta \Rightarrow \sigma^2 = 2kT\xi
\end{equation}
This is the simplest example of the \emph{Fluctuation-Dissipation theorem}\index{theorem!fluctuation-dissipation}. The $-\xi \dot{x}$ term enforces an arrow of time on the system pushing it towards lower energy. The variance of $f$ is then fixed to restore the time reversal symmetry. For the rest of the course we shall write;
\begin{equation}
f = \sqrt{2kT\xi}\Lambda
\end{equation}
where now $\left< \Lambda(t)\Lambda(t\pr) \right> = \delta(t - t\pr)$ describes the case of white noise.
\end{definitionbox}
\subsubsection{The Overdamped Limit}
In the overdamped limit, $m\ddot{x} \ll \xi\dot{x}$ so the Langevin equation becomes;
\begin{align*}
\xi \dot{x} &= -\nabla V + \sqrt{2kT\xi}\Lambda \\
\Rightarrow \dot{x} &= -\tilde{M}\nabla V + \sqrt{2kT\tilde{M}}\Lambda
\end{align*}
where $\tilde{M} = 1/\xi$ is the \emph{mobility}\index{mobility} (velocity per unit force). We want to move from the Langevin equation to the diffusion equation. We define $P(x, t)$ as the probability density of being at $x$ at time $t$. Then in a time $\Delta t$, the probability of moving a distance $\Delta x$ is;\footnotemark
\footnotetext{
We consider the path $x(t)$ that moves by $\Delta x$ in $\Delta t$, then the forward probability is exactly $P_{\Delta t}(\Delta x)$ and is given by;
\begin{align*}
\mathbb{P}_F[x(t) \mapsto (x + \Delta x)(t + \Delta t)] &= \mathcal{N}\exp\left(-\frac{1}{2\sigma^2}\int_{t}^{t + \Delta t}{\upd{t}(\xi \dot{x} + \nabla V)^2}\right) \\
&\sim \mathcal{N}\exp\left(-\frac{1}{2(2kT\xi)}\frac{1}{\Delta t}\left((\xi \dot{x} + \nabla V)\Delta t\right)^2\right) \\
&\sim \mathcal{N}\exp\left(-\frac{1}{4kT\xi\Delta t}(\xi \Delta x + \nabla V \Delta t)^2\right)
\end{align*}
Note we have also assumed that $\xi$ and $\tilde{M}$ are independent of $x$.
}
\begin{equation}
P_{\Delta t}(\Delta x) = \mathcal{N}\exp\left(-\frac{1}{4kT\Delta t}\left(\xi \Delta x + \nabla V \Delta t\right)^2\right)
\end{equation}
We redefine $u \coloneqq \Delta x$ and then $P_{\Delta t}(u) \coloneqq W(u, x)$. We can then read of the mean and variance;
\begin{align}
\left< u \right> &= \int{\upd{u} uW(u, x)} = -\frac{\nabla V}{\xi}\Delta t \\ 
\left< u^2 \right> &= \int{\upd{u}u^2W(u, x)} = \frac{2kT}{\xi}\Delta t + \mO(\Delta t^2) \\
\nabla\left< u \right> &= \int{\upd{u}u\nabla W} = \nabla\int{\upd{u}uW} = -\frac{\nabla^2 V}{\xi}\Delta t
\end{align}
We want to find an expression for $\dot{P}(x, t)$ so we write;
\begin{equation*}
P(x, t + \Delta t) = \int{\upd{u}P(x - u, t)W(u, x - u)}
\end{equation*}
i.e. the probability starting from a point $(x - u)$ at time $t$ of moving to $x$ in time $\Delta t$ summed over all possible starting positions $(x - u)$. We can expand this as a Taylor series;
\begin{equation*}
P(x, t + \Delta t) = \int{\upd{u}(P(x, t) - u\nabla P + \tfrac{1}{2}u^2\nabla^2 P)(W(u, x) - u\nabla W + \tfrac{1}{2}u^2\nabla^2 W)}
\end{equation*}
where the derivatives are taken with respect to $x$. Using the integrals above and gathering the terms we find;
\begin{equation}
\dot{P} = D\nabla^2 P + \tilde{M}\nabla(P\nabla V), \qquad D = \frac{kT}{\xi}, \quad \tilde{M} = \frac{1}{\xi}
\end{equation}
This is a differential equation for the probability density of finding the particle with the relation between $D$ and $\tilde{M}$ another realisation of the FDT\index{FDT}. We can write this as;
\begin{equation*}
\dot{P}_1 = -\nabla \cdot \vec{J}_1
\end{equation*}
where $\vec{J}_1$ is a probability current;
\begin{equation}
\vec{J}_1 = -P_1 D\nabla\left(\log P_1 + \beta V\right) = -P_1 \tilde{M}\nabla \mu(x)
\end{equation}
where $\mu(x) = kT \log P_1 + V$.\footnote{i.e. the chemical potential of a particle in a potential $V(x)$} This then measures the response of a particle to a chemical potential gradient with a strength governed by the mobility\index{mobility} $\tilde{M}$. Note that;
\begin{enumerate}
\item It is deterministic for $P_1$
\item It contains the same information as the Langevin\index{equation!Langevin} which is stochastic for $x(t)$
\end{enumerate}
\subsection{From One Particle to Many}
Suppose we have $N$ non-interacting colloids\index{colloid} in $V(x)$ with a coarse-grained density field $\rho(\vec{x}, t)$. Now we know that $\left< \rho \right> = NP_1$ and $\left< \dot{\rho} \right> = N\dot{P}_1$. Then we should still be able to write;
\begin{equation}
\dot{\rho} = -\nabla \cdot \vec{J}
\end{equation}
where $\left< \vec{J} \right> = -\rho\tilde{M}\nabla \mu$ and $\mu = kT \log \rho + V(x)$. If we have interactions we should set $\mu = \delta F / \delta \rho$, although this is ultimately a phenomenological choice. 

\paraskip
Now, we should not just set $\vec{J} = \left< \vec{J} \right>$ as this neglects fluctuations in the particle current. If we did we would have a \emph{hydrodynamic level}\index{hydrodynamic level} description of the system\footnote{Note that this is qualitatively different to the one particle case. There we had a deterministic evolution that had already taken account of the noise in the Langevin equation. Here we have done no such thing, and still need to account for the fluctuations in the underlying particle density field.} for $\dot{\rho}$ which would evolve to a mean field solution ($\mu$ a constant and $F[\rho]$ at a minimum). Instead of a mean field solution, we would like to have the Boltzmann\index{Boltzmann distribution} on the space of field configurations;
\begin{equation}
\mathcal{P}[\rho] = \mathcal{N}\exp\left(-\beta F[\rho]\right)
\end{equation}
which arises from a Langevin equation for $\rho$ (playing the role of $x(t)$). This requires;
\begin{align}
\dot{\rho} &= -\nabla \cdot \vec{J} \\
\vec{J} &= -\rho \tilde{M} \nabla\mu + \vec{j}
\end{align} 
where now $\vec{j}$ is a stochastic noise, without which would return to the mean field solution.\footnote{Note that we shouldn't add another term in the $\dot{\rho}$ equation; $\rho$ is still a conserved quantity and thus should have some form of current divergence.} We want to implement this with constant $M = \rho \tilde{M}$ to avoid multiplicative noise\index{multiplicative noise}. This is called the \emph{collective mobility}\index{mobility!collective}. We will assume that $\vec{j}$ is Gaussian and white\footnotemark, so;
\footnotetext{
So that we have the following;
\begin{equation*}
\left< j_k(\vec{r}, t) j_l(\vec{r}\pr, t\pr) \right> = \sigma^2 \delta_{kl} \delta(\vec{r} - \vec{r}\pr)\delta(t - t\pr)
\end{equation*}
}
\begin{equation}
\mathbb{P}[\vec{j}] = \mathcal{N}\exp\left(-\frac{1}{2\sigma^2}\int_{t_1}^{t_2}{\upd{t}\int{\upd{\vec{r}}\abs{\vec{j}(\vec{r}, t)}^2}}\right)
\end{equation}  
We should then repeat the detailed balance\index{detailed balance} arguments to find the variance $\sigma^2$. Starting now from $\vec{J} + M\nabla \mu = \vec{j}$, we see that;
\begin{equation}
\mathbb{P}_F[\vec{J}(\vec{r}, t)] = \mathcal{N}\exp\left(-\frac{1}{2\sigma^2}\int_{t_0}^{t_1}{\upd{t}\int{\upd{\vec{r}}\abs{\vec{J} + M\nabla\mu}^2}}\right)
\end{equation} 
where $\mu = \delta F[\rho]/\delta \rho$, and the same for $\mathbb{P}_B$ with $\vec{J}\mapsto -\vec{J}$. So we find that;
\begin{align*}
\log\left(\frac{\mathbb{P}_F}{\mathbb{P}_B}\right) &= -\frac{2M}{\sigma^2}\int_{t_0}^{t_1}{\upd{t}\int{\upd{\vec{r}}\vec{J}\cdot\nabla\mu}} \\
&= \frac{2M}{\sigma^2}\int_{t_0}^{t_1}{\upd{t}\int{\upd{\vec{r}}(\nabla \cdot \vec{J})\mu}} \\
&= \frac{2M}{\sigma^2}\int_{t_0}^{t_1}{\upd{t}\int{\upd{\vec{r}}\left(-\dot{\rho}\frac{\delta F}{\delta \rho}\right)}} \\
&= -\frac{2M}{\sigma^2}\int_{t_0}^{t_1}{\upd{t}\frac{\ud F[\rho]}{\ud t}} \\
\Rightarrow \log\left(\frac{\mathbb{P}_F}{\mathbb{P}_B}\right) &= -\frac{2M}{\sigma^2}\set{F_2 - F_1}
\end{align*}
But detailed balance requires that $e^{-\beta F_1}\mathbb{P}_F = e^{-\beta F_2}\mathbb{P}_B$ which constrains the variance;
\begin{equation}
\frac{2M}{\sigma^2} = \beta \Rightarrow \sigma^2 = \frac{2M}{\beta}
\end{equation}
Then the final many body Langevin equations are;
\begin{align}
\dot{\rho} &= - \nabla \cdot \vec{J} \\
\vec{J} &= -M \nabla\left(\frac{\delta F}{\delta \rho}\right) = \sqrt{2kTM}\vec{\Lambda}
\end{align}
where $\vec{\Lambda}$ is spatiotemporal unit white noise;
\begin{equation*}
\left< \Lambda_i(\vec{r}, t) \Lambda_j(\vec{r}\pr, t\pr) \right> = \delta_{ij}\delta(\vec{r} - \vec{r}\pr)\delta(t - t\pr)
\end{equation*}
The same form of equation then holds for any diffusive conserved scalar, for example $\phi(\vec{r}, t)$ in the binary fluid.
\subsection{Dynamics of Fluctuations}
\subsubsection{Model B\index{model B}}
This is a model of pure diffusion with no fluid flow, for a scalar composition field, $\phi(\vec{r})$
\begin{align}
\dot{\phi} &= -\nabla \cdot \vec{J} \\
\vec{J} &= -M \nabla \mu + \sqrt{2 k T M} \vec{\Lambda}
\end{align}
where $\mu = \delta F/\delta \phi$ and;
\begin{equation*}
F[\phi] = \int{\upd{\vec{r}}\set{\frac{a}{2}\phi^2 + \frac{b}{4}\phi^4 + \frac{k}{2}(\nabla \phi)^2}}
\end{equation*}
So $\mu = a\phi + b\phi^3 - k\nabla^2 \phi$. This produces exactly the pictures shown in \autoref{fig:binfluid} and \autoref{fig:mftbin}. We define the following regions;
\begin{enumerate}
\item $\abs{\bar{\phi}} > \phi_B$: uniform phase which is globally stable
\item $\abs{\bar{\phi}} < \phi_S$: $f^{\prime\prime}(\phi) < 0$ here and it is locally unstable (spinodal\index{spinodal})
\item $\phi_S < \abs{\bar{\phi}} < \phi_B$: locally stable but globally not, this leads to nucleation\index{nucleation} and growth
\end{enumerate}
\subsubsection*{Region 1}
In this region we will have small fluctuations around $\phi(\vec{r}) = \bar{\phi}$. So write $\phi = \bar{\phi} + \tilde{\phi}(\vec{r})$, then;
\begin{align*}
\mu &= \frac{\del f}{\del \phi} - k \nabla^2 \phi = f\pr(\phi) + \tilde{\phi}f^{\prime\prime}(\bar{\phi}) - k\nabla^2 \tilde{\phi} \\
\Rightarrow \dot{\phi} &= -\nabla \cdot \vec{J}, \quad \vec{J} = -M \nabla\left(f^{\prime\prime}\cdot \tilde{\phi} - k \nabla^2 \tilde{\phi}\right) + \sqrt{2kTM}\vec{\Lambda}
\end{align*}
We can take the Fourier transform;
\begin{equation*}
\dot{\phi}_{\vec{q}} = -Mq^2 \left(f^{\prime\prime} + kq^2\right)\phi_{\vec{q}} + i\sqrt{2kTM}\vec{q}\cdot\vec{\Lambda}
\end{equation*}
This allows us to define a \emph{dynamic structure factor};
\begin{align*}
S(q, t) &= \left< \phi_{\vec{q}}(0)\phi_{\vec{q}}(t) \right> = S(q)e^{-r(q)t} \\
r(q) &= Mq^2\left(f^{\prime\prime} + kq^2\right)
\end{align*}
where $S(q)$ is the equilibrium and equal time static correlator; $S(q) = kT/(f^{\prime\prime} + kq^2)$. This basically states that ultimately, time correlators should decay at a characteristic rate from their initial equal time fluctuations (they should not have an everlasting effect). 
\subsubsection*{Region 2}
We have the same expression for the Fourier modes, but now $f^{\prime\prime} < 0$. Thus the fluctuations are unstable for $q^2 < -f^{\prime\prime}/k$. Averaging the equation about we find;
\begin{equation*}
\left< \dot{\phi}_{\vec{q}} \right> = -r(q)\left< \phi_{\vec{q}} \right>
\end{equation*}
with $-r(q) > 0$ at $q = \sqrt{-f^{\prime\prime}/k}$, i.e. now defines a growth rate. This implies that from an initial, noisy state $\phi_{\vec{q}}(0)$, we see;
\begin{equation*}
\left< \phi_{\vec{q}}(t) \right> = \phi_{\vec{q}}(0)e^{-r(q)t}
\end{equation*}
We will see an amplification of the initial noise. Furthermore, not that this implies that noise at late time becomes unimportant as it will have an amplitude that is negligible compared to the growth modes. We still have $r(q) = Mq^2(f^{\prime\prime} + kq^2)$ as shown in \autoref{fig:sqt}. 
\begin{mygraphic}{tpscm/sqt}{0.4}{We see that there is a minimum value $q_{\star}$ that sets a lengthscale for the system}{sqt}\end{mygraphic}
The minimum $q_{\star}$ is defined by $r\pr(q_{\star}) = 0$, so the fastest growing modes are near $q_{\star} = \sqrt{-f^{\prime\prime}/2k}$. Now define the equal time non-equilibrium structure factor. This measure the equal time variance in the fluctuations of each mode;
\begin{equation}
S_q(t) = \left< \phi_{\vec{q}}(t)\phi_{-\vec{q}}(t) \right> \sim S_q(0)e^{-2r(q)t}
\end{equation}
\begin{mygraphic}{tpscm/sqt2}{0.6}{The equal time non-equilibrium structure factor. Note that the initially noisy profile is smoother by the evolution and picks out the preferred mode $q_{\star}$.}{sqt2}\end{mygraphic}
The picking out of a preferred mode is illustrated in \autoref{fig:sqt2}. This process is called spinodal decomposition\index{spinodal decomposition}; we observe growths of random structures with a characteristic length scale $L \sim \pi/q_{\star}$. At intermediate $t$ the quartic term kicks in leading to an `effective' $f^{\prime\prime}$;
\begin{equation*}
\bar{f}^{\prime\prime} = f^{\prime\prime} + \frac{3b}{(2\pi)^d}\int^{q_{\text{max}}}{\upd{^d q}S_{q}(t)}
\end{equation*}
Then, at least heuristically, $f^{\prime\prime}$ becomes less negative as fluctuations grow. So $q_{\star}$ moves to smaller $q$. Thus $L(t) \sim q_{\star}^{-1}$ increases leading to \emph{domain growth}\index{domain growth}. At late stages we have regions of $\phi \sim \pm \phi_B$ so we are almost in equilibrium locally, but we still have $L = L(t)$. The driving force for the system at this point is then the reduction of interface area;
\begin{equation*}
\frac{F}{V} \sim \frac{\sigma A(t)}{V} \sim \frac{\sigma}{L(t)}
\end{equation*}
where $\sigma$ is the equilibrium interfacial tension\index{interfacial tension}.
\begin{mygraphic}{tpscm/domgrowth}{1.0}{An illustration of the concept of domain growth to produce phase separation, from a paper by \href{https://lucris.lub.lu.se/ws/files/4018932/4580825.pdf}{Wittkowski}}{domgrowth}\end{mygraphic}
\subsubsection*{Region 3}
Suppose now we start at a point $\bar{\phi} = -\phi_B + \delta$, then this is locally stable since $r(q) > 0$, but globally unstable. It is always energetically preferable to separate into phases. But, only noise can overcome the nucleation barrier\index{nucleation barrier}. We want to answer the question as to what the most likely path is? Formally we should seek the most likely trajectory from;
\begin{equation*}
\mathbb{P}[\phi(\vec{r}, t)] = \mathcal{N}\exp\left(-\frac{\beta}{4M}\int{\abs{J + M\nabla \mu}}^2 \ud\vec{r}\ud t\right)
\end{equation*}
which is known as the \emph{instanton path}. We follow a more basic route, illustrated in \autoref{fig:nucl};
\begin{mygraphic}{tpscm/nucl}{0.9}{ The process of nucleation and growth. Random noise overcomes the free energy barrier due to interface area and leads to a growing region of phase separation.}{nucl}\end{mygraphic}
We understand this as follows;
\begin{itemize}
\item Free energetically, it is always worse to have a greater area, so we have a contribution to the free energy of the form $4\pi \sigma R^2$
\item On the other hand, the free energy is lowered by splitting into phases, so we have a decrease due to the bulk term $-\tfrac{4}{3}\pi R^3$
\end{itemize}
Plotting this, we see that there is a critical free energy $F_{\star}$ at some critical radius $R_{\star}$ that must be overcome due to noise. Thus, the noise induced rate of the process is proportional to the Boltzmann distribution $e^{-\beta F_{\star}}$. 
\begin{mygraphic}{tpscm/nuclbarr}{0.6}{The competition between nucleation area and volume leads to a critical free energy $F_{\star}$ that must be overcome by random noise.}{tpscm/nuclbarr}\end{mygraphic}
\subsubsection{Droplet Growth (Coarsening)\index{coarsening}}
Suppose we have two bulk phases with compositions $\phi_1, \phi_2$ separated by a curved interface. In the bulk we have $\mu = \delta f/\delta \phi$ with a pressure;\footnote{This expression is motivated by thermodynamics, but also arises as the isotropic part of the stress tensor.}
\begin{equation*}
\Pi = \mu \phi - f
\end{equation*}
We consider two distinct cases illustrated in \autoref{fig:flatdroplet};
\begin{enumerate}
\item \emph{Flat Interface:} $R \rightarrow \infty$, here we simply have;
\begin{align*}
\mu_1^{\text{bulk}} &= \mu_2^{\text{bulk}} \\
\Pi_1^{\text{bulk}} &= \Pi_2^{\text{bulk}}
\end{align*}
We see in the figure that we have the same intercept and the same slope.
\item \emph{Droplet:} finite $R$. We consider the force balance of the upper hemisphere;
\begin{equation*}
\Pi_2 = \Pi_1 + \underbrace{\frac{\sigma}{R}(d - 1)}_{\text{Laplace pressure}}
\end{equation*}
Then in a static equilibrium, we have $\vec{J} = 0$ so that $\nabla \mu = 0 \Rightarrow \mu_1 = \mu_2 = \mu$.
\end{enumerate}
\begin{mygraphic}{tpscm/flatdroplet}{0.9}{The two cases considered; the flat interface (left) where there is no pressure difference between the two bulk phases, and the finite, curved case (right) where the pressure difference drives a potential gradient.}{flatdroplet}\end{mygraphic}
It is the second case we are most interested in. Note that $f(\phi_B) = f(-\phi_B)$ so that;
\begin{align*}
f_1 &= f(-\phi_B + \delta_1) = f(-\phi_B) + \frac{1}{2}f^{\prime\prime}(\phi_B)\delta_1^2 + \cdots \\
f_1 &= f(-\phi_B + \delta_1) = f(\phi_B) + \frac{1}{2}f^{\prime\prime}(\phi_B)\delta_2^2 + \cdots 
\end{align*}
So we see that $\mu_1 = \alpha \delta_1$, $\mu_2 = \alpha \delta_2$ so that $\delta_1 = \delta_2 = \delta$. Hence, in global equilibrium, the pressure is given by;
\begin{align*}
\Pi_1 &= \mu\phi - f = \alpha \delta (- \phi_B - \delta) - \frac{\alpha}{2}\delta^2 \\
\Pi_2 &= \alpha \phi_B \delta \\
\Rightarrow \Pi_1 - \Pi_2 &= -2\alpha \phi_B \delta \equiv -(d - 1)\frac{\sigma}{R} \\
\Rightarrow \delta &= \frac{1}{2\alpha \phi_B}(d - 1) \frac{\sigma}{R}
\end{align*}
This is the state of coexistence across a curved interface with $\phi = \pm \phi_B + \delta$. We now want to apply this at the hydrodynamic MF level for many droplets. Suppose we have a droplet in $d = 3$ in a global composition $\phi = -\phi_B + \epsilon$. $\epsilon$ is known as the \emph{supersaturation}\index{supersaturation}. The supersaturation is set by $\left< R \right>$ of the other droplets, so it is a property of the external environment. Now let the composition external to the droplet be;
\begin{equation}
\phi(\vec{r}) = -\phi_B + \tilde{\phi}(\vec{r}) \Rightarrow \tilde{\phi}(\infty) = \epsilon
\end{equation}
Then at least locally we must have $\tilde{\phi}(R^{+}) = \delta(R)$ where $R$ is the radius of the droplet. Model B\index{model B} says that;
\begin{equation*}
\dot{\phi} = -\nabla \cdot \vec{J}, \quad \vec{J} = -M\nabla \mu = -M \alpha \nabla \tilde{\phi}(\vec{r})
\end{equation*}
Assuming that we have some sort of quasistatic behaviour when $\dot{\phi} = 0 \Rightarrow \nabla^2 \tilde{\phi} = 0$. We then need to solve this with the boundary conditions $\tilde{\phi}(\infty) = \epsilon$ and $\tilde{\phi}(R^{+}) = \delta(R)$
\begin{equation}
\Rightarrow \tilde{\phi} = \epsilon + (\delta - \epsilon)\frac{R}{r}
\end{equation}
This in turn leads to a current just outside the droplet;
\begin{align*}
\vec{J}(R^{+}) &= -\alpha M \left.\frac{\del \tilde{\phi}}{\del r}\right|_{R^+} \\
&= \alpha M (\delta - \epsilon) \left.\frac{R}{r^2}\right|_{r = R^{+}} \\
&= \frac{\alpha M (\delta - \epsilon)}{R}
\end{align*}
We see that there is a discontinuous jump in $\phi$ across the interface; $\Delta \phi = 2\phi_B$ so applying mass conservation;
\begin{equation*}
2\phi_B \dot{R} = - J \Rightarrow \dot{R} = \frac{1}{2\phi_B}\left(\frac{\alpha M}{R}\left(\epsilon - \delta(R)\right)\right)
\end{equation*}
In 3D $\delta = \sigma/\alpha R\phi_B$ and we can plot the phase diagram as shown in \autoref{fig:bubble};
\begin{mygraphic}{tpscm/bubble}{0.8}{We see that $R^{\star}$ is an unstable fixed point; colloquially, large droplets grow, and small droplets shrink. This can be understood as there being a current/fluid flow into/out of the droplet causing it to change size.}{bubble}\end{mygraphic}
Now, assume a scaling ansatz that there is only one relevant length scale set by the mean droplet size $\bar{R}$. Then;
\begin{align*}
\dot{\bar{R}} &\sim \frac{1}{2\phi_B}\frac{\alpha M}{\bar{R}}\left(\epsilon - \delta(\bar{R})\right) \\
\Rightarrow \dot{\bar{R}} &\sim \frac{M\sigma}{\phi_B^2 \bar{R}^2} \Rightarrow \bar{R}^{3} \sim \frac{M\sigma t}{\phi_B^2}
\end{align*}
So the typical droplet size scales as $\bar{R} \propto R^{\star}t^{1/3}$. Similarly $\epsilon \propto t^{-1/3}$ so the supersaturation relaxes to the equilibrium value of $\epsilon = 0$. This explains the process of nucleation droplets evolving to a flat interface separating phases. This mechanism is known as \emph{Ostwald ripening}\index{Ostwald ripening}, which is a diffusive coarsening process. 

\paraskip
We actually see the same scaling for non-droplet geometries e.g. spinodal decomposition at late times: diffusive flux moves stuff from areas of high curvature to low curvature, causing the domains to flatten, we find that;
\begin{equation*}
L(t) \sim \left(\frac{M\sigma}{\phi_B^2}t\right)^{1/3}
\end{equation*}
\subsubsection*{Preventing the Ostwald process}
How do we prevent this process? This is essential for storage and stability of emulsions\index{emulsion}. We usually add a trapped species that in insoluble in the continuous phase e.g. polymers/salt. Then (in the ideal case);
\begin{equation*}
\Pi_2 - \Pi_1 = \frac{2\sigma}{R} + \frac{NkT}{\tfrac{4}{3}\pi R^3}
\end{equation*}
We must still have $\mu_1 = \mu_2$ across the interface, leading to the boundary condition at $R^{+}$ of:
\begin{equation*}
\tilde{\phi}(R^{+}) = \frac{\sigma}{\alpha R\phi_B} - \frac{3NkT}{8\alpha\phi_B \pi R^3}
\end{equation*}
This in turn implies that;
\begin{equation}
\dot{R} = \frac{1}{2\phi_B}\frac{\alpha M}{R}\left(\epsilon - \frac{\sigma}{\alpha \phi_B R} + \frac{3NkT}{8\alpha \phi_B \pi R^3}\right)
\end{equation}
This introduces a new, stable fixed point at $R_s < R^{\star}$ where there is a balance between osmotic and Laplace pressure terms. There is then no driving force for coarsening and the emulsion is stabilised.
\subsubsection{Model H}
We want to move on from pure diffusion to consider flow also;
\begin{align}
\underbrace{\dot{\phi} + \vec{v}\cdot \nabla \phi}_{\text{advection}} &= -\nabla \cdot \vec{J} \\
\vec{J} &= -M\frac{\delta F}{\delta \phi} + \sqrt{2kTM}\vec{\Lambda}
\end{align}
We also need an equation for $\vec{v}$ which we get by imposing incompressibility $\nabla\cdot\vec{v} = 0$ and assuming it satisfies a Cauchy equation with stress tensor $\Sigma_{\text{tot}}$;
\begin{equation}
\phi(\dot{\vec{v}} + \vec{v}\cdot\nabla \vec{v}) = \nabla \cdot \Sigma_{\text{tot}}
\end{equation}
We have $\Sigma_{\text{tot}} = \Sigma^{p} + \Sigma^{\eta} + \Sigma^{\phi} + \Sigma^{N}$ where the terms are explained as follows;
\begin{itemize}
\item $\Sigma^P_{ij} = -P\delta_{ij}$ simply acts as a Lagrange multiplier on $\nabla\cdot\vec{v} = 0$
\item $\Sigma^{\eta}_{ij} = \eta(\nabla_i v_j + \nabla_j v_i)$ is the viscous stress for a fixed $\eta$
\item $\Sigma^{\phi}_{ij} = -\Pi \delta_{ij} - \kappa (\nabla_i \phi)(\nabla_j \phi)$ where $\Pi = \phi \mu - \mathbb{F}$ is the order parameter stress
\item $\Sigma^{N}_{ij}$ is the noise stress satisfying
\begin{equation*}
\left< \Sigma_{ij}^{N}(\vec{r}, t)\Sigma_{kl}^{N}(\vec{r}\pr, t\pr) \right> = 2kT\eta \left(\delta_{ik}\delta_{jl} + \delta_{il}\delta_{jk}- \frac{2}{3}\delta_{ij}\delta_{kl}\right)\delta(\vec{r} - \vec{r}\pr) \delta(t - t\pr)
\end{equation*} 
which describes white noise from the $\eta$ term in line with the FDT\index{FDT}. 
\end{itemize}
As such we find that;
\begin{equation}
\nabla \cdot \Sigma_{\text{tot}} = -\nabla P + \eta \nabla^2 v - \phi \nabla \mu + \nabla \cdot \Sigma^{N}
\end{equation}
\begin{definitionbox}[Model H]
With these definitions, Model H becomes;
\begin{align}
\dot{\phi} + \vec{v}\cdot\nabla \phi &= -\nabla \cdot \vec{J} \\
\vec{J} &= -M\nabla \mu + \sqrt{2lTM}\vec{\Lambda} \\
\nabla \cdot \vec{v} &= 0 \\
\rho(\dot{\vec{v}} + \vec{v}\cdot\nabla\vec{v}) &= \eta \nabla^2 v - \nabla P - \phi \nabla \mu + \nabla \cdot \Sigma^{N}
\end{align}
\end{definitionbox}
What new physics does this give us?
\begin{enumerate}
\item $-\phi \nabla \mu$ describes a deterministic fluid flow
\item The noise stress drives a random flow
\item Both of these flows advect
\end{enumerate}
The first and the last of these give enhanced coarsening of bicontinuous states, but have no effect on droplet states. This is because for an isolated droplet, they cannot drive a radial flow. The second and third then give the fluid a random velocity leading to Brownian motion of the droplets. Now we have;
\begin{equation*}
\left< r^2 \right> \propto Dt, \quad D = \frac{kT}{4\pi \eta R}
\end{equation*}
which ensures that the droplets can now meet and coalesce, for which separate measures are required to prevent e.g. charges on the surface.
\subsubsection*{Coalescence}
Again, we assume there is only one length scale $\bar{R}(t)$ which governs size and separation of droplets. Then we can attribute a collision time;
\begin{equation*}
\bar{R}^2 \sim D\Delta t \Rightarrow \Delta t \sim \frac{\eta}{kT}\bar{R}^3
\end{equation*}
Each collision doubles the volume so $\bar{R}\mapsto 2^{1/3}\bar{R}$ so that in a time $\Delta t$;
\begin{equation*}
\frac{\Delta \log \bar{R}}{\Delta t} \sim \frac{\log 2}{3}\frac{kT}{\eta \bar{R}^3}
\end{equation*}
Thus, at least scaling wise we can write;
\begin{equation*}
\frac{\ud \log \bar{R}}{\ud t} = \frac{1}{\bar{R}}\dot{\bar{R}} \sim \frac{kT}{\eta \bar{R}^3} \Rightarrow \bar{R}(t) \sim \left(\frac{kT}{\eta}t\right)^{1/3}
\end{equation*}
which describes \emph{diffusion limited coalescence}\index{diffusion limited coalescence} - a different process to Ostwald ripening. For example, trapped species have no effect. 
\subsubsection*{Domain Growth via Flow}
Let $L(t)$ be the domain size in a bicontinuous structure. Again, assume that there is only this single length scale. Typically, if we identify $v \sim \dot{L}$ and the Laplace pressure by $\sigma/L$ we see that the more curved domains shrink and the flat ones grow in an attempt to minimise the interfacial area. In terms of the evolution, $\Sigma^N$ only plays a role at early times in setting up the random initial conditions. What then is the scaling for $L(t)$? From;
\begin{equation*}
\rho(\dot{\vec{v}} + \vec{v}\cdot \nabla \vec{v}) = \eta \nabla^2 \vec{v} - \nabla P - \phi \nabla\mu
\end{equation*}
we can extract the following scaling `equation';
\begin{equation*}
\rho \ddot{L} + \rho \frac{\dot{L}^2}{L} = \eta \frac{\dot{L}}{L^2} + \frac{\sigma}{L^2}
\end{equation*}
We have three dimensional parameters in the problem $\set{\rho, \eta, \sigma}$\footnote{Note that $a, b, k \in \mathbb{F}$ do enter the scaling, but only in the combination determined by $\sigma$, not via $\xi$. This is simply the statement that the interfacial width is very small at anything but early times.} In terms of units;
\begin{equation*}
\rho = ML^{-3}, \quad \eta = ML^{-1}T^{-1}, \quad \sigma = MT^{-2}
\end{equation*}
from which we can construct;
\begin{equation}
L_0 = \frac{\eta^2}{\rho\sigma}, \qquad t_0 = \frac{\eta^3}{\rho \sigma^2}
\end{equation}
Thus we find that by a dimensional analysis argument;
\begin{equation}
L(t) = L_0 f(t/t_0)
\end{equation}
Plugging this into our scaling equation we find that;
\begin{equation*}
\alpha f^{\prime\prime} + \beta \frac{1}{f}(f\pr)^2 = \gamma \frac{1}{f^2}f\pr + \delta \frac{1}{f^2}
\end{equation*}
where $\alpha, \beta, \gamma, \delta = \mO(1)$. We have two distinct regimes;
\begin{enumerate}
\item \emph{Early Times:}the LHS governs the acceleration of the fluid, at small times, this is negligible so we set up the balance;
\begin{equation*}
\gamma \frac{f\pr}{f^2} + \delta \frac{1}{f^2} = 0 \Rightarrow f\pr = \text{const.} 
\end{equation*} 
So we find that $L/L_0 = \text{const.}\cdot t/t_0$ and hence at early times;
\begin{equation}
L(t) \propto \frac{\sigma}{\eta}t
\end{equation}
This is known as the \emph{viscous-hydrodynamic regime}.
\item \emph{Late Times:} or equivalently, at large $f$ where $L \gg L_0$ we assume a power law behaviour $f \propto x^{y}$, so that;
\begin{equation*}
\alpha x^{y - 2} + \beta x^{y - 2} = \gamma x^{-y - 1} + \delta x^{-2y}
\end{equation*}
In this region, interfacial energy is converted into kinetic energy via the Laplace pressure. Only later is this dissipated by the viscosity, so we neglect the $\gamma$ term. Then balancing exponents gives $y = 2/3$, giving;
\begin{equation}
L(t) \propto \left(\frac{\sigma}{\rho}\right)^{1/3} t^{2/3}
\end{equation}
This is known as the \emph{inertial hydrodynamic regime}. 
\end{enumerate}
Thus, in $d = 3$ we find the behaviour as in \autoref{fig:hydro};
\begin{mygraphic}{tpscm/hydro}{0.6}{We have found the relevant scaling regimes for $L(t)$. The first region in the plot is where the Ostwald process can take over as $\dot{L}_{\text{diff}} \sim \dot{L}_{\text{VH}}$.}{hydro}\end{mygraphic}
\subsubsection*{Droplet vs Bicontinuous}
Consider quenching the system from a high temperature at some fixed global composition $\bar{\phi}$. Then the phase separation is governed by;
\begin{equation*}
-\phi_B V_1 + \phi_B V_2 = \bar{\phi}V, \quad V_1 + V_2 = V
\end{equation*}
which allows us to write down an expression for the phase space volume, $\psi = V_2 / V$;
\begin{equation}
\psi = \frac{1}{2}\left(1 + \frac{\bar{\phi}}{\phi_B}\right)
\end{equation}
which tells us how much space is taken up by a given phase. In $d = 3$ we have the following approximate regions;
\begin{itemize}
\item $0.4 < \psi < 0.6$ gives a bicontinuous state
\item $\psi < 0.3, \psi > 0.7$ leads to droplets
\item $0.3 < \psi < 0.4, 0.6 < \psi < 0.7$ initially forms a bicontinuous structure, but then depercolates
\end{itemize}
For droplets of size $R$ and separation $L$, $\psi \sim R^3 / L^3$ so $R \sim L$ as $\psi$ is a constant. In $d = 2$ we have different behaviour due to the topology. Only at $\psi = 0.5$ do we have bicontinuity, which arises due to the topological requirement that there should a continuous path from one side of the fluid to the other. Thus, droplet states are generic in $2$D.
\subsection{Liquid-Crystal Hydrodynamics}
We start by working at the hydrodynamic level, neglecting noise, and consider the two following cases;
\begin{enumerate}
\item \emph{Polar liquid crystals:}\index{liquid crystal!polar} here the order parameter is $\vec{p}$ where we orientational but not positional order
\item \emph{Nematic liquid crystals:}\index{liquid crystal!nematic} as before the order parameter is the tensor valued $Q$
\end{enumerate}
\subsubsection{Polar Liquid Crystal}
We consider the free energy;
\begin{equation}
F = \int{\upd{\vec{r}}\frac{a}{2}\abs{\vec{p}}^2 + \frac{b}{4}\abs{p}^4 + \frac{k}{2}(\nabla_i p_j)(\nabla_i p_j)} \equiv \int{\upd{\vec{r}}\mathbb{F}}
\end{equation}
We make the following notes;
\begin{itemize}
\item $F[\vec{p}] = F[-\vec{p}]$, i.e. there are no cubic terms
\item The linear term would be an external magnetic field
\item The $k$ term penalises splay/twist/bend roughly equally
\end{itemize}
In this case, unlike Models B and H, $\vec{p}$ is not conserved, instead (without flow) it is described by;
\begin{equation}
\dot{\vec{p}} = -\Gamma \vec{h}
\end{equation}
Here $\vec{h} = \delta F/\delta \vec{p}(\vec{r})$ is known as the \emph{molecular field}\index{molecular field}. We want to add flow to this description which is done schematically via promoting the derivative to include advection;
\begin{equation*}
\frac{D\vec{p}}{Dt} = -\Gamma \vec{h}
\end{equation*}
where for example, a scalar would have;
\begin{equation*}
\frac{D\phi}{Dt} = \dot{\phi} + \vec{v}\cdot \nabla \phi
\end{equation*}
For $\vec{p}$ we have translation as in the scalar case $\Delta \vec{p} = \vec{v}\cdot \nabla \vec{p}\,\, \Delta t$ but we also have co-rotation with the fluid;
\begin{equation*}
\Delta \vec{p} = \omega \wedge \vec{p} \,\, \Delta t
\end{equation*}
where $\omega_i = \tfrac{1}{2}\epsilon_{ijk}\Omega_{jk}$ and $\Omega_{jk} = \tfrac{1}{2}(\nabla_j v_k - \nabla_k v_j)$ is the angular velocity of the fluid. This part must be present as we could for example rotate the whole system as a rigid body. We also need a third term. This is given by;
\begin{equation*}
\Delta \vec{p} = -\xi D \cdot \vec{p} \Delta t, \quad D_{ij} = \frac{1}{2}\left(\nabla_i v_j + \nabla_j v_i\right)
\end{equation*}
which says that in a general flow, $\vec{p}$ typically aligns along streamlines. All together then we have;
\begin{equation*}
\frac{D\vec{p}}{D t} = (\del_t + \vec{v}\cdot\nabla)\vec{p} + \Omega \cdot \vec{p} - \xi D\cdot \vec{p}
\end{equation*}
$\xi$ is a molecular parameter that depends on the type of liquid crystal. For example a polymer might stretch under flow. The hydrodynamic level equations are then;
\begin{equation}
\frac{D\vec{p}}{Dt} = -\Gamma \vec{h}, \quad \vec{h} = \frac{\delta F}{\delta \vec{p}}
\end{equation}
The equation for $\vec{v}$ is then given by;
\begin{equation}
\rho(\dot{\vec{v}} + \vec{v}\cdot \nabla \vec{v}) = \eta \nabla^2 \vec{v} - \nabla P + \nabla \cdot \Sigma^P
\end{equation}
where $\Sigma^{p}$ is the order parameter stress. Now, the free energy change in advective elastic distortion is;
\begin{equation*}
\delta F = \int{\upd{\vec{r}}\frac{\delta F}{\delta \vec{p}}\cdot \dot{\vec{p}}\Delta t} = \Delta t \int{\vec{h}\cdot\dot{\vec{p}}}
\end{equation*}
where we have taken;
\begin{equation*}
\frac{D\vec{p}}{Dt} = 0 \iff \dot{\vec{p}} = -\vec{v}\cdot \nabla \vec{p} - \Omega \cdot \vec{p} + \xi D\cdot \vec{p}
\end{equation*}
In other words, we are considering pure advection with no relaxation. This must obey;
\begin{equation*}
\delta F = \int{\upd{\vec{r}}\Sigma_{ij}^{p}\nabla_i u_j(\vec{r})}
\end{equation*}
From which we find that;
\begin{equation*}
\Sigma^{P}_{ij} = \Sigma_{ij}^{1} + \Sigma_{ij}^{2} + \Sigma_{ij}^{3}
\end{equation*}
where;
\begin{itemize}
\item $\nabla_i \Sigma_{ij}^{1} = -p_k \nabla_j h_k$ - the counterpart to $-\phi \nabla \mu$ 
\item $\Sigma^{2}_{ij} = \tfrac{1}{2}\left(p_i h_j - p_j h_i\right)$ - from $\Omega \cdot \vec{p}$
\item $\Sigma^{3}_{ij} = \tfrac{\xi}{2}(p_i h_j + p_j h_i)$ from $\xi D\cdot \vec{p}$
\end{itemize}
Thus we find the hydrodynamic level equations for a polar liquid crystal;
\begin{align}
(\del_t + \vec{v}\cdot \nabla)\vec{p} &= -\Omega \cdot \vec{p} + \xi D\cdot\vec{p} - \Gamma \vec{h} \\
\rho(\del_t + \vec{v}\cdot \nabla)\vec{v} &= \eta \nabla^2 \vec{v} - \nabla P + \nabla \cdot \Sigma^{p}
\end{align}
\subsubsection{Nematic Liquid Crystal\index{liquid crystal!nematic}}
The equations for the nematic liquid crystal are broadly of the same form with some additional complications that ensure $\tr Q = 0$ for all $t$. The exact form of these relations is unimportant, but the coarsening dynamics are interesting. Suppose we start in an initial state above the critical temperature where $Q = 0$ then a shallow quench\index{quench!shallow} has $f^{\prime\prime}(0) > 0$ ensuring that there is a nucleation barrier, as illustrated in \autoref{fig:nematicpt}. On the other hand, a deep quench\index{quench!deep} induces spinodal-like stability. Now;
\begin{equation*}
\frac{DQ}{Dt} = -\Gamma H, \qquad H = \frac{\del f}{\del Q} - k \nabla^2 Q
\end{equation*}
where $f$ is the local free energy density. We have a non-conserver order parameter, so we have maximum growth $r(0)$ at scale $q = 0$ i.e. ``order can develop from anywhere''. This means that on a time scale $r(0)^{-1}$ we \emph{locally} have;
\begin{equation*}
Q \mapsto \thrbythr{\lambda&&}{&-\lambda/2&}{&&-\lambda/2}
\end{equation*}
i.e. $Q = \hat{\lambda}(n_i n_j - \tfrac{1}{3}\delta_{ij})$ where importantly $\vec{n} = \vec{n}(\vec{r})$. The early stage coarsening fixes the $\hat{\lambda}$ then late stage coarsening equilibrates $\vec{n}(\vec{r})$ driven by the gradient term in $\mathbb{F}$;
\begin{equation*}
\mathbb{F}_{el} = \frac{k}{2}(\nabla \cdot Q)^2
\end{equation*}
In comparison with Models B and H, the domain walls that form are an example of a \emph{topological defect}.\index{topological defect}
\begin{definitionbox}[Topological Defect]
A topological defect is a a structure of reduced dimensionality $D < d$ that connects different ground states ($\equiv$ minima of $F$). They cannot be removed by any local process. An example is the the domain wall seen in the Landau-Ginzburg $\phi^4$ theory.
\end{definitionbox}
\subsubsection{Topological Defects in Nematic Liquid Crystals}
Start in 2D by first noticing that there are no line defects;
\begin{mygraphic}{tpscm/nolined}{0.6}{There are no line defects in 2D, we can simply locally relax the system to induce a continuous transition}{nolined}\end{mygraphic}
On the other hand we can have point defects, as illustrated in \autoref{fig:nematicd}. Here we see a difference between polar molecules which have a definite orientation and nematic liquid crystals. In the former case, we must go round the point defect twice in order to return to the same configuration. This is then said to have \emph{topological charge}\index{topological charge} $\abs{q} = 1$. In the nematic case, we have $\abs{q} = \tfrac{1}{2}$ where the sign is determined by the sense of the rotation. In general, these topological defects can dissociate i.e. $q = 1 \rightarrow 2 \cdot q = \tfrac{1}{2}$. This occurs for energetic reasons and implies that we should find lots of small ($q$ small) defects rather than fewer large ones.
\begin{mygraphic}{tpscm/nematicd}{0.5}{Examples of topological defects in nematic liquid crystals with the topological charge $k$ noted above each figure. Taken from a paper on the \href{Orientation of topological defects in 2D nematic liquid crystals}{https://arxiv.org/pdf/1706.05065.pdf}}{nematicd}\end{mygraphic}
Now consider this in the case of coarsening in 2D. In the early stages we have $p \rightarrow 2\lambda (\vec{n}\vec{n} - \mathbb{I})$ where $\vec{n} = \vec{n}(\vec{r})$ is random. This is uniform subject to topological defects (TDs). However, TDs cannot be ironed our locally. Then we have a late stage process, where opposite charges attract and annihilate. Consider the elastic free energy of an isolated defect;
\begin{equation*}
\frac{\tilde{k}}{2}\abs{(\nabla \cdot \vec{n})\vec{n} + \vec{n}\cdot \nabla \vec{n}}^2, \quad \tilde{k} = k\lambda^2
\end{equation*}
Dimensionally in the neighbourhood of a defect, $\nabla \sim 1/r$ we find that the energy scales as;
\begin{equation*}
E \sim \tilde{k} \int{\upd{\vec{r}}\frac{1}{r^2}} \sim \tilde{k}\log\frac{L}{r_0}
\end{equation*}
where $L$ is the mean spacing, and $r_0$ is the core radius within with $\lambda$ is not constant. We see that this $E$ is coulombic in nature with a force proportional to $1/R$ in $2$D. Making the identification $v_{\text{\footnotesize{defect}}} \propto \text{force}$ we find that;
\begin{equation*}
\dot{R} \propto R^{-1} \Rightarrow \dot{L} \propto L^{-1} \Rightarrow L(t) \propto t^{1/2}
\end{equation*}
which is the same scaling law as nematic coarsening in 3D.
\subsubsection{Topological Defects in 3D}
In $3$D, the point defects discussed above become line defects (consider the line coming out of the page in the diagrams in \autoref{fig:nematicd}). The interaction force remains proportional to $1/R$ and $L(t) \propto t^{1/2}$. The difference arises however in that in $3$D, the following are in fact the same topologically.
\begin{mygraphic}{tpscm/3ddefect}{0.6}{The above are topologically equivalent in $3$D with a mapping $+q \rightarrow -q$ via a local $180$ degree, out-of-plane rotation about an axis that points upwards in the plane of the defect.}{3ddefect}\end{mygraphic}
\subsubsection{Homotopy Theory}\index{homotopy theory}
Suppose we have an order parameter space $\mM$ which is the space of group states. For an $n$-dimensional spin vector $\mM = S^{n - 1}$, the surface of an $n$-dimensional sphere. For nematics in $2$D this is a unit circle with opposite points identified, $P_1$. In $3$D, this is the $2$-sphere with antipodal points identified, $P_2$. Now, suppose that we specify the order parameter on a domain $\mD$ (loop/surface) that encloses the defect. This creates a mapping from $\mD \rightarrow \mM$.
\begin{mygraphic}{tpscm/dmap}{0.8}{The image of $\mathcal{D}$ is a closed contour in $\mM$.}{dmap}\end{mygraphic}
Two mappings $f_0$ and $f_1$ are \emph{homotopic}\index{homotopic} if they are continuously deformable into each other. Defects lie in the same \emph{homotopy class}\index{homotopy class} if maps for all $\mD$ enclosing them are continuously deformable. For example in $2$D nematics we can wind round $n$ times to give a topological charge of $q = \pm n/2$.
\subsubsection*{Homotopy Group}\index{group!homotopy}
Consider two defects with loops $\mD_1$ and $\mD_2$ surrounding them. Then we can consider a contour $\mD_3 = \mD_1 + \mD_2$ surrounding both defects. Then topological defects\index{topological defect} form a group under addition. For example, in $2$D, we have $q_{1 + 2} = q_1 + q_2$. This is the \emph{fundamental homotopy group}\index{group!fundamental homotopy}, $\pi_1(\mM)$. We can consider higher dimensional surfaces enclosing the defect, described by the group $\pi_n(\mM)$. 

\paraskip
Now, what about $3$D nematics? There are only two possibilities for the maps on $P_2$. Firstly we have the identity map where the loop closes on the same point. The other possibility is that the loop closes (since antipodal points are identified) on the antipodal point. Thus the fundamental homotopy group is $\pi_1(\mM) = \ZZ_2$. The outcome of this discussion is that $3$D nematics support only a single defect; a dislocation line. Furthermore, two dislocations can annihilate or appear out of nothing and any even number can eventually disappear. There is another case to consider; that of a dislocation loop\index{dislocation loop} where at each point on the loop, we appear to have one of the defects listed above. 
\begin{mygraphic}{tpscm/loopdis}{0.7}{The two line defects manifest as loop dislocations.}{loopdis}\end{mygraphic}
Note that the distant patterns of these $3$D loop dislocations are for example, in the first case a $3$D hedgehog where $\vec{n} = \vec{r}$. This is the scenario when the loop is shrunk to zero, leaving a point defect. We need to surround this by a $2$-surface to understand the defect, so they lie in the second homotopy group $\pi_2(\mM)$.
\subsubsection{Topological Defects in Polar Liquid Crystals}\index{liquid crystal!polar}
Now we have a vector order parameter $\vec{p}$, so in $2$D our manifold is just a circle $\mM = S^1$, so the paths start and end at the same point as there is no identification. Then we can simply wrap round $n$ times leading to topological charges\index{charge!topological} $q = \pm n$, where each charge is in a separate homotopy class\index{homotopy class}.
\begin{mygraphic}{tpscm/pol2d}{0.7}{Possible topological defects for a polar liquid crystal in $2$D}{pol2d}\end{mygraphic}
In $3$D on the other hand, $\mM = S^2$ and all paths can be deformed to a point, so there are no topological defects\footnote{Or more precisely, there are no \emph{line} defects since the fundamental homotopy group is trivial.}. There is however point defects for a polar liquid crystal (the $3$D ``hedgehog'' and ``anti-hedgehog'') with charges $q = \pm 1$. These lie in the second homotopy group $\pi_2(S^2)$ as we must surround them by a $2$-surface.
\newpage
\section{Active Soft Matter}
In this section, the guiding example is that of mobile, self-propelled particles;
\begin{enumerate}
\item Micro-organisms including bacteria and algae which have a preferred swimming direction
\item Synthetic microswimmers for example the asymmetric \ce{AgPt} molecules in the presence of \ce{H2O2}. This leads to asymmetric catalysis of \ce{2 H2O2 -> 2 H2O + O2}, which in turn results in gradients in the reagents/products across the particle surface chemistry $\Rightarrow$ motion.
\end{enumerate}
Now, the micromechanics of these systems rely on the continuous conversion of fuel to motion $\iff$ continuous entropy production. Thus the microscopic dynamics have no Newtonian/Lagrangian equaitons and are as such \emph{not} time reversible. This has the consequence that for steady state behaviour, there is \emph{no Boltzmann distribution}, and the principle of detailed balance\index{detailed balance} no longer holds. This means that we are allowed to have a steady state particle flux e.g. circular flux which changes direction on time reversal. As an example, consider bacteria in a microfluidic enclosure as in \autoref{fig:microfluidenc}.
\begin{mygraphic}{tpscm/microfluidenc}{0.7}{In a colloidal system, our intuition may be that we still see this sort of circular behaviour, however this is in fact not the case. The steady state would still be uniform.}{microfluidenc}\end{mygraphic}
This system is an example of the Ratchet theorem where if we have broken time reversal symmetry pathwise, along with breaking spatial symmetry (parity/reflection), then $\vec{J} \neq 0$.

\paraskip 
As before we have two options for how to develop models. We could explicitly coarse grain the ``microscopic'' dynamics with rules governing the particle motion, leading to PDEs for the coarse-grained order parameters. Or, as we shall do, we can start with models of passive soft matter e.g. Models B and H, or liquid crystal hydrodynamics and add minimal terms to explictly break time reversal.

\subsection{Active Model B} 
As an example of this second viewpoint, consider a symmetric scalar field $\phi$ which is diffusive and exhibits phase separation behaviour $\Rightarrow$ should consider Model B.
\begin{align*}
\dot{\phi} &= -\nabla \cdot \vec{J} \\
\vec{J} &= -\nabla \tilde{\mu} + \sqrt{2D}\Lambda \\
F &= \int{\upd{\vec{r}}\frac{a}{2}\phi^2 + \frac{b}{4}\phi^4 + \frac{k}{2}(\nabla \phi)^2} \\
\tilde{\mu} &= \frac{\delta F}{\delta \phi} + \underbrace{\lambda (\nabla \phi)^2}_{\text{new term}}
\end{align*}
This is known as \emph{Active Model B}\index{active model B}. We make the following observations;
\begin{enumerate}
\item $(\nabla \phi)^2$ is not the functional derivative of any $F$ which breaks the free energy structure, so;
\begin{equation*}
\frac{\mathbb{P}_F}{\mathbb{P}_B} \neq \exp\left(-\beta(F_2 - F_1)\right)
\end{equation*}
for any functional $F$. So time reversal symmetry is broken.
\item Note that;
\begin{equation*}
g(\phi) = \frac{\delta}{\delta \phi}\int{\upd{\vec{r}}\int^{\phi}{\upd{u}g(u)}}
\end{equation*}
so polynomial terms never break time reversal symmetry in the scalar case. So gradient terms are indeed needed.\footnote{Compare this to the polar/nematic liquid crystal case}
\item Active Model B is agnostic about the cause of phase separation for $a < 0$;
\begin{itemize}
\item We could have attractive interactions, in a very similar way to the normal Model B
\item On the other hand, we could have repulsive interactions combined with mobility leading to pairwise jamming (think of two oppositely ``charged'' molecules propelling themselves towards each other leading to some pseudo-stationary dynamics). This is known as MIPS (mobility induced phase separation)\index{MIPS}. 
\end{itemize}
\item In terms of the dynamics of coarsening, we find that $L(t) \sim t^{1/3}$ and an Ostwald-like process remains manifest.
\item On the other hand, the coexistence conditions are altered, in the passive case for a \emph{flat} interface, we had simply $\mu_1 = \mu_2$ and $p_1 = (\mu \phi - f)_1 = (\mu \phi - f)_2 = p_2$, and saw a flat tangent with the same intercept (c.f. \autoref{fig:flatdroplet}). In contrast, the bulk phase separation in the active case exhibits an ``honorary Laplace pressure'' where there is a difference in intercept, $\Delta$, so that;
\begin{equation*}
\mu_1 = \mu_2, \quad (\mu \phi - f)_1 = (\mu \phi - f)_2 + \Delta(\lambda)
\end{equation*}
This is found by solving $\vec{J} = 0$ so that;
\begin{equation*}
\tilde{\mu} = \frac{\del f}{\del \phi} - k \nabla^2 \phi + \lambda (\nabla \phi)^2 = \text{const.}
\end{equation*}
\item We have a further extension to the model, known as AMB+;
\begin{align*}
\dot{\phi} &= - \nabla \cdot \vec{J} \\
\vec{J} &= -\nabla \tilde{\mu} + \sqrt{2D}\lambda \\
\tilde{\mu} &= \frac{\delta F}{\delta \phi} + \lambda (\nabla \phi^2) + \xi (\nabla^2 \phi) \nabla \phi
\end{align*}
where we have now included the term proportional to $\xi$. In $1$D this is in fact the same as the $\lambda$ term, however in general it changes the dynamics. For more information, see \href{https://arxiv.org/pdf/1801.07687.pdf}{this arXiv article}.
\end{enumerate}
\subsection{Active Liquid Crystals}
\subsubsection{Active Polar Liquid Crystals}
Suppose we have a polar order parameter $\vec{p}$, which in the simplest case is just relaxational (when $\vec{v} = 0$). Then at the passive hydrodynamic level, we have $\dot{\vec{p}} = -\Gamma \vec{h}$ where $\vec{h} = \delta F/\delta \vec{p}$;
\begin{equation*}
F = \int{\upd{\vec{r}}\frac{a}{2}\abs{\vec{p}}^2 + \frac{b}{4}\abs{\vec{p}}^4 + \frac{k}{2}(\nabla_\alpha p_\beta)(\nabla_\alpha p_\beta)}
\end{equation*}
As before, we could add gradient terms to $\dot{\vec{p}} = \cdots$ that are incompatible with $F$. However there is something lower order in the polar case that is not possible with scalars. Consider a swimming rod that translates along it's own direction at some characteristic speed $w_0$. Then a meso-averaging process leads to the order parameter developing a self advective velocity $w\vec{p}$ where $w \propto w_0$. This leads to;
\begin{equation*}
\dot{\vec{p}} + w \vec{p}\cdot\nabla \vec{p} = -\Gamma \vec{h}
\end{equation*}
which resembles the Navier-Stokes instability. Crucially though, $\vec{p}$ does not change under time reversal, it is an order parameter, so this term manifestly breaks time reversal symmetry. We couple this to $\vec{v}$ via;
\begin{equation}
\frac{D\vec{p}}{Dt} = \left(\frac{\del}{\del t} + \vec{v}\cdot\nabla\right)\vec{p} + \Omega \cdot\vec{p} - \xi D\cdot\vec{p} + w\vec{p}\cdot\nabla \vec{p} = -\Gamma \vec{h}
\end{equation}
We also need an equation for the velocity;
\begin{equation}
(\del_t + \vec{v}\cdot\nabla)\vec{v} = \eta \nabla^2 \vec{v} - \nabla P + \nabla\cdot\Sigma^{(p)}
\end{equation}
where the final term is the order parameter stress. This develops a new term due to the active motion;
\begin{equation*}
(\nabla \cdot \Sigma^{(p)})_i = -p_i \nabla_j h_j + \nabla_j\left(\frac{1}{2}(p_j h_i - p_i h_j)\right) + \frac{\xi}{2}\nabla_j\left(\frac{1}{2}(p_j h_i + p_i h_j)\right) + \nabla \cdot\Sigma^{A}
\end{equation*}
To lowest order this active stress $\Sigma^{A} = \zeta p_i p_j$ which is a new mechanical term incompatible with $F$. So we find;
\begin{equation}
\nabla \cdot\Sigma^{A} = \vec{p}(\nabla \cdot \vec{p})
\end{equation}
which acts as an effective body force in the Navier-Stokes equation\index{equation!Navier-Stokes}. It is ambivalent to the orientation of $\vec{p}$, and leads to spontaneous flow, symmetry breaking and macroscopic fluxes. This leads to ``turbulence''\footnote{Not in the same sense as the Navier-Stokes equation.} and chaos at high $w, \zeta$.
\subsubsection{Active Nematic Liquid Crystals}
Since we have a tensor order parameter, there is no self advection. As discussed before, hydrodynamically we have;
\begin{align*}
\frac{DQ}{Dt} &= (\del_t + \vec{v}\cdot\nabla)Q + S(Q, K, \xi) = -\Gamma H, \quad H = \frac{\delta F}{\delta Q} \\
S &= - (\Omega \cdot Q - Q\cdot\Omega) - \xi \left(D \cdot Q + Q\cdot D\right) + 2\tilde{\zeta}\left(Q + \frac{\mathbb{I}}{d}\right)\tr(Q\cdot K)
\end{align*} 
Thus, at least hydrodynamically, this remains unchanged by the active motion of the nematic. The leading order time reversal symmetry breaking term is $\Sigma^{A} = \zeta Q$ which breaks the free energy structure. To see this, consider a uniform nematic then $\Sigma_{\text{passive}} = 0$ since the stress arises only due to elastic distortions of the crystal. But $\Sigma^{A} = Q$ arises from arbitrarily local flow fields around each particle. The effective body force density $\vec{f} = \nabla\cdot\Sigma^{A} = \zeta \nabla \cdot Q$;
\begin{equation}
\Rightarrow \vec{f} \sim \zeta \lambda (\nabla \cdot\vec{n})\vec{n}
\end{equation}
for $Q = \tfrac{3}{2}\lambda(n_i n_j - \delta_{ij}/3)$. So suppose we have a splay for the director field\index{director field}, $\vec{n}$, then the effective body force acts to the right. This produces the sort of behaviour seen in \autoref{fig:splay};
\begin{mygraphic}{tpscm/splay}{0.7}{Under a perturbation of the uniform state into one with local splaying, we see that the effective body force is destabilising for $\zeta > 0$}{splay}\end{mygraphic}
Less obviously, the bend is also destabilised for $\zeta < 0$ so that active nematics are \emph{never stable} in a uniform state for a large system limit. Typically then, you end up with spontaneous flow. The sign of the activity parameter\index{activity parameter}, $\zeta$, depends on the local flow patter. For nematics, we have the possibilities shown in \autoref{fig:actpara};
\begin{mygraphic}{tpscm/actpara}{0.6}{The different possible flow fields around a nematic molecule}{actpara}\end{mygraphic}
Now consider a shear flow; a rod-like object will tend to align along the extension axis, with a viscous stress given by;
\begin{equation*}
\Sigma^{\eta} = \eta g \twobytwo{0}{1}{1}{0}
\end{equation*}
whilst the active stress is given by;\footnote{In the case that $\vec{n}$ is oriented at exactly $45$ degrees.}
\begin{equation*}
\Sigma^{A} \propto \zeta \lambda \left(\vec{n}\vec{n} - \frac{\mathbb{I}}{d}\right) = \zeta \lambda\twobytwo{0}{1}{1}{0}
\end{equation*}
Now consider the total stress, we see the following profile;
\begin{mygraphic}{tpscm/totstress}{0.6}{The total stress $\Sigma_{\text{tot}} = \Sigma^{\eta} + \Sigma^{A}$. We see that in the contractile case, there is no pathological behaviour, the system does not move unless a finite stress is applied. On the other hand, even for a viscous fluid, in the extensile case, there is some $g^{\star}$ for which there is a spontaneous flow.}{totstress}\end{mygraphic}
In consequence we can have something akin to a phase separation with a velocity profile akin to that shown in \autoref{fig:velprofile}, leading to a spontaneous flow, or ``superfluidity'' in the absence of applied stress.\footnote{Consider for example flow inside a circular torus, in the absence of active motion, there could be no net flow.}
\begin{mygraphic}{tpscm/velprofile}{0.6}{The phenomenon of spontaneous flow/superfluidity in an active nematic liquid crystal.}{velprofile}\end{mygraphic}
\subsection{Defect Motion in Active Nematics}
In $2$D we have the two principal defects as in \autoref{fig:3ddefect}. The rightmost one cannot have an unbalanced stress simply by symmetry, however the $q = +1/2$ defect can. In this case the $x$-component of the effective force density is given by;
\begin{equation*}
(\nabla \cdot \Sigma^{A})_x = f_x = \left(\zeta \nabla \cdot Q\right)_x \sim(\del_x n_x)n_x \sim \zeta
\end{equation*}
So we see that the $q = +1/2$ defects themselves move as quasiparticles, downwards in the case of an extensile activity parameter, and upwards in the contractile case (referring to the orientation in the figure referenced). Ultimately, the outcome of this is self-sustaining turbulent motion and defect pairs $q = \pm 1/2$ are formed locally all the time. The $q = -1/2$ stay put, by the $q = +1/2$ self propel against the elastic attraction. This results in a rapid dissociation of defect pairs. For references on this, see \href{https://www.nature.com/articles/nature11591}{Sanchez} or \href{http://pubs.rsc.org/en/content/articlepdf/2017/sm/c6sm02310j}{Shendruk}.
















%\end{multicols*}